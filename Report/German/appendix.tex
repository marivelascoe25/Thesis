% eigene Verzeichnisse für den Anhang
% Quelle: https://komascript.de/node/2115
\DeclareNewTOC[%
  owner=\jobname,
  listname={Anhang}, % Titel des Verzeichnisses
]{atoc} % Dateierweiterung (a=appendix, toc=table of contents)

\DeclareNewTOC[%
  owner=\jobname,
  listname={Abbildungen im Anhang}, % Titel des Verzeichnisses
]{alof} % Dateierweiterung (a=appendix, lot=list of figures)

\DeclareNewTOC[%
  owner=\jobname,
  listname={Tabellen im Anhang}, % Titel des Verzeichnisses
]{alot} % Dateierweiterung (a=appendix, lot=list of tables)

% Befehle zum Umschalten der von TOC, LOF, LOT zu verwendenden Dateierweiterungen
\makeatletter
\newcommand*{\useappendixtocs}{%
  \renewcommand*{\ext@toc}{atoc}%
  \scr@ifundefinedorrelax{hypersetup}{}{% damit es auch ohne hyperref funktioniert
    \hypersetup{bookmarkstype=atoc}
  }%
  % Falls die Abbildungen und Tabellen im Anhang mit im Abbildungs- bzw.
  % Tabellenverzeichnis des Hauptteils erscheinen sollen, sind die folgenden
  % zwei Zeilen auszukommentieren!
  \renewcommand*{\ext@figure}{alof}%
  \renewcommand*{\ext@table}{alot}%
}
\newcommand*{\usestandardtocs}{%
  \renewcommand*{\ext@toc}{toc}%
  \scr@ifundefinedorrelax{hypersetup}{}{% damit es auch ohne hyperref funktioniert
    \hypersetup{bookmarkstype=toc}%
  }%
  \renewcommand*{\ext@figure}{lof}%
  \renewcommand*{\ext@table}{lot}%
}
\scr@ifundefinedorrelax{ext@toc}{%
  \newcommand*{\ext@toc}{toc}
  \renewcommand{\addtocentrydefault}[3]{%
    \expandafter\tocbasic@addxcontentsline\expandafter{\ext@toc}{#1}{#2}{#3}%
  }
}{}
\makeatother

% anpassen des appendix-Befehls, so dass Inhaltsverzeichnis des Anhangs an dieser Stelle eingefügt wird
\usepackage{xpatch}
\xapptocmd\appendix{%
  \phantomsection
  \addcontentsline{toc}{section}{\appendixname}
  \useappendixtocs
  \listofatocs % Inhaltsverzeichnis
  \listofalofs % Verzeichnis der Abbildungen im Anhang
  \listofalots % Verzeichnis der Tabellen im Anhang
}{}{}

