% Dies ist eine Latex-Vorlage fuer Arbeiten am IFTE der TU Dresden
% Die gezeigten Kopfzeilen etc. sind Layoutvorschläge, keine Richtlinien!

\documentclass[%
  paper=a4,
  fontsize=12pt,
  twoside=false,
  headsepline=true,
  numbers=noenddot]{scrartcl}
\usepackage{ifthen}

\newboolean{Diplomarbeit}
\setboolean{Diplomarbeit}{false}

\usepackage{setspace} % Änderung des Zeilenabstands

% Font
\usepackage[T1]{fontenc} % Korrekte Trennung von Wörtern mit Umlauten
\usepackage{lmodern}

% Layout (siehe layout.tex)
\usepackage[left=3cm,bottom=3cm,right=3cm,top=3cm]{geometry}

% Kopfzeilen
\usepackage[autooneside=false]{scrlayer-scrpage}
\pagestyle{scrheadings}
\automark[subsection]{section}
% Kopfzeile rechtsbündig
\ohead{\headmark}
\chead[]{}

% Sections erscheinen immer auf einer neuen Seite ohne Kopfzeile
\makeatletter
\newcommand*{\the@orig@section}{}
\let\the@orig@section\section
\renewcommand*{\section}{%
  \clearpage
  \thispagestyle{plain}%
  \the@orig@section
}
\makeatother


% Sprache
\usepackage[ngerman,silent]{babel} % für deutsches Inhaltsverzeichnis etc.
\usepackage[utf8]{inputenc} % Unterstützung für deutsche Umlaute in der TeX-Datei

% Grafiken
\usepackage{graphicx}
\usepackage{pdfpages} % Einbinden von PDF-Dateien
\usepackage{caption}
\usepackage{float} % Positionierung von Bildern "hier" [H]

% Symbole
\usepackage{textcomp}
\usepackage{latexsym}
\usepackage{marvosym}
\usepackage{amssymb}

% Mathematik
\usepackage{amsmath}
\usepackage{bm}                 % fette Symbole im Mathemodus
\usepackage{ziffer}             % kein Abstand nach Dezimalkomma
\usepackage[locale=DE]{siunitx} % Angabe von SI-Einheiten

% Tabellen
\usepackage{multirow}
\usepackage{longtable}
\usepackage{array} % Better tabulars
\newcommand{\PreserveBackslash}[1]{\let\temp=\\#1\let\\=\temp}
\usepackage[NewCommands]{ragged2e} % Better raggedleft formating
\newcolumntype{v}[1]{>{\PreserveBackslash{\raggedright}\hspace{0pt}}p{#1}}

% Anhang (siehe appendix.tex)
% eigene Verzeichnisse für den Anhang
% Quelle: https://komascript.de/node/2115
\DeclareNewTOC[%
  owner=\jobname,
  listname={Anhang}, % Titel des Verzeichnisses
]{atoc} % Dateierweiterung (a=appendix, toc=table of contents)

\DeclareNewTOC[%
  owner=\jobname,
  listname={Abbildungen im Anhang}, % Titel des Verzeichnisses
]{alof} % Dateierweiterung (a=appendix, lot=list of figures)

\DeclareNewTOC[%
  owner=\jobname,
  listname={Tabellen im Anhang}, % Titel des Verzeichnisses
]{alot} % Dateierweiterung (a=appendix, lot=list of tables)

% Befehle zum Umschalten der von TOC, LOF, LOT zu verwendenden Dateierweiterungen
\makeatletter
\newcommand*{\useappendixtocs}{%
  \renewcommand*{\ext@toc}{atoc}%
  \scr@ifundefinedorrelax{hypersetup}{}{% damit es auch ohne hyperref funktioniert
    \hypersetup{bookmarkstype=atoc}
  }%
  % Falls die Abbildungen und Tabellen im Anhang mit im Abbildungs- bzw.
  % Tabellenverzeichnis des Hauptteils erscheinen sollen, sind die folgenden
  % zwei Zeilen auszukommentieren!
  \renewcommand*{\ext@figure}{alof}%
  \renewcommand*{\ext@table}{alot}%
}
\newcommand*{\usestandardtocs}{%
  \renewcommand*{\ext@toc}{toc}%
  \scr@ifundefinedorrelax{hypersetup}{}{% damit es auch ohne hyperref funktioniert
    \hypersetup{bookmarkstype=toc}%
  }%
  \renewcommand*{\ext@figure}{lof}%
  \renewcommand*{\ext@table}{lot}%
}
\scr@ifundefinedorrelax{ext@toc}{%
  \newcommand*{\ext@toc}{toc}
  \renewcommand{\addtocentrydefault}[3]{%
    \expandafter\tocbasic@addxcontentsline\expandafter{\ext@toc}{#1}{#2}{#3}%
  }
}{}
\makeatother

% anpassen des appendix-Befehls, so dass Inhaltsverzeichnis des Anhangs an dieser Stelle eingefügt wird
\usepackage{xpatch}
\xapptocmd\appendix{%
  \phantomsection
  \addcontentsline{toc}{section}{\appendixname}
  \useappendixtocs
  \listofatocs % Inhaltsverzeichnis
  \listofalofs % Verzeichnis der Abbildungen im Anhang
  \listofalots % Verzeichnis der Tabellen im Anhang
}{}{}



% Links und Bookmarks im PDF
\usepackage[%
  final,
  bookmarks=true,
  bookmarksopen,
  bookmarksnumbered=true,
  linktoc=all,
  hidelinks % Links nicht hervorheben
]{hyperref}
\usepackage{bookmark}

% \cref gibt automatisch die Art des Ziels einer Referenz an, z.B. \cref{fig4} -> "Abbildung 4"
% Man spart sich also das Tippen von "Abbildung".
\usepackage[noabbrev]{cleveref} % unbedingt nach hyperref laden

\begin{document}

% Verwendung von kleingeschriebenen römischen Zahlen (i,ii,...) für alle Inhalte vor Abschnitt 1
\pagenumbering{roman}

% 1,5-facher Zeilenabstand
\onehalfspacing

% Titelseite
% *****************************************************************************

\thispagestyle{empty}
\begin{center}
\Large \textbf{TECHNISCHE UNIVERSITÄT DRESDEN \\ \vspace{1cm}
FACULTY OF ELECTRICAL\\
AND COMPUTER ENGINEERING\\ \vspace{1cm}
Dresden Integrated Center for \\
Applied Physics and Photonic Materials\\ \vspace{2cm}
{\Huge \ifthenelse{\boolean{Diplomarbeit}}{DIPLOMARBEIT}{Project work report}}}\\ \vspace{1cm}
\end{center}
\begin{flushleft}
\begin{tabular}{ll}\\
Topic: &  Doping p-type Organic Electrochemical Transistors\\ 
& to tune threshold voltage\\ \ifthenelse{\boolean{Diplomarbeit}}{}{\vspace{4cm}}
&\\

Presented by: & Marielena Velasco Enriquez\\
Department: & Nanoelectronic Systems \hspace{1cm} EMM Nano\\\\\ifthenelse{\boolean{Diplomarbeit}}{}{\\}

\end{tabular}
\end{flushleft}

\ifthenelse{\boolean{Diplomarbeit}}{
\begin{center}
zum\\
Erlangen des akademischen Grades\\
\vspace{0.4cm}
{\Huge \textbf{DIPLOMINGENIEUR}}\\
\vspace{0.25cm}
(Dipl.-Ing.)\\
\end{center}
}{}


\begin{flushleft}
\begin{tabular}{ll}\\
Supervisor: & Dr.rer.nat.habil. Hans Kleemann\\
&\\
Principal Investigator: & Prof. Karl Leo\\
&\\
Day of submission: & 20.02.2023\\

\end{tabular}
\end{flushleft}

 % siehe titlepage.tex

% Aufgabenstellung
% *****************************************************************************

\cleardoublepage
\thispagestyle{empty} % keine Kopf- und Fußzeile (Seitenzahl) anzeigen

\noindent Anstatt dieser Seite ist die \underline{originale}, vom verantwortlichen Hochschullehrer unterzeichnete Aufgabenstellung einzubinden. Die weiteren abzugebenden Versionen der Diplomarbeit enthalten eine Kopie der Aufgabenstellung.

Das Binden der \ifthenelse{\boolean{Diplomarbeit}}{Diplomarbeit}{Studienarbeit} hat so zu erfolgen, dass ein nachträglicher Seitenaustausch nicht möglich ist (keine Spiralbindung).

\cleardoublepage

%\includepdf{Aufgabe.pdf}
%\includepdf[pages1-4]{ddd.pdf}

% Selbstständigkeitserklärung
% *****************************************************************************

\section*{Selbstständigkeitserklärung}

Hiermit versichere ich, dass ich die vorliegende Arbeit ohne unzulässige Hilfe Dritter und ohne Benutzung anderer als der angegebenen Hilfsmittel angefertigt habe; die aus fremden Quellen direkt oder indirekt übernommenen Gedanken sind als solche kenntlich gemacht. Bei der Auswahl und Auswertung des Materials sowie bei der Herstellung des Manuskripts habe ich Unterstützungsleistungen von folgenden Personen erhalten:

\vspace*{1cm}
\hspace*{2cm}xxxx

\hspace*{2cm}xxxx

\hspace*{2cm}xxxx

\hspace*{2cm}xxxx
\vspace*{1cm}

\noindent Weitere Personen waren an der geistigen Herstellung der vorliegenden Arbeit nicht beteiligt.
Mir ist bekannt, dass die Nichteinhaltung dieser Erklärung  zum nachträglichen
\ifthenelse{\boolean{Diplomarbeit}}{Entzug des Diplomabschlusses}{Aberkennen der Prüfungsleistung} führen kann.

\vspace*{2cm}
\noindent Dresden, den xx.xx.xx \hspace{3cm} Unterschrift: \\

\cleardoublepage

% Kurzfassung
% *****************************************************************************

\section*{Kurzfassung}

Die Kurzfassung in deutscher und englischer Sprache hat das Ziel, dem Leser Inhalt, Bedeutung und Einordnung der Arbeit in das Fachgebiet auf einer gesonderten Seite vorzustellen.

\vspace*{2cm}

\makeatletter
\the@orig@section*{Abstract} % in layout.tex wurde der \section-Befehl umdefiniert; wir brauchen hier den originalen \section-Befehl, da der Abstract sonst auf einer neuen Seite erscheint
\makeatother

\cleardoublepage


% Inhaltsverzeichnis
% *****************************************************************************

\tableofcontents


% Abbildungsverzeichnis
% *****************************************************************************

\listoffigures


% Tabellenverzeichnis
% *****************************************************************************

\listoftables

% Zeichen, Benennungen und Einheiten
% *****************************************************************************

\section*{Zeichen, Benennungen und Einheiten}

\subsection*{Abkürzungen}

\begin{tabular}{ll}
ADC&			Analog-Digital-Umsetzer\\
DAC&			Digital-Analog-Umsetzer\\
GPIO&			Programmierbarer Eingabe-/Ausgabekontakt eines Mikrocontrollers\\
\end{tabular}

\subsection*{Formelzeichen}

\begin{tabular}{ll}
  $A_\text{Pel}$ & Peltierfläche in \si{\milli\metre\squared}\\
  $A_\text{s}$ & Spannungsquerschnitt in \si{\milli\metre\squared}\\
\end{tabular}

% Umstellung der Seitennummerierung von römisch auf arabisch
\cleardoublepage
\pagenumbering{arabic}

% Einleitung
% *****************************************************************************

\section{Einleitung}

Der Textteil beginnt mit der Einleitung. In dieser  sind die Zusammenhänge, die zur
Formulierung der Aufgabe geführt haben (Motivation) klar und auch für einen "`Außenstehendean"' verständlich darzustellen. Ebenfalls sind hier die Einordnung der Arbeit in das
Fachgebiet und die Nutzungsmöglichkeiten der Ergebnisse zu beschreiben. Wenn die
Bearbeitung in einer Firma erfolgte, ist diese zu nennen.


% Stand der Technik
% *****************************************************************************
\section{Stand der Technik}
%\input{StandDerTechnik}

Beispiel für physikalische Größen: \SI{120.2}{\milli\watt}. Bei Verwendung von \texttt{siunitx} wird ein Dezimalpunkt automatisch durch ein Dezimalkomma ersetzt.

Das Paket \texttt{ziffer} sorgt weiterhin dafür, dass auch im normalen Mathemodus nach einem Dezimalkomma kein zusätzlicher Abstand eingefügt wird: $123,45$.

Über den Befehl \texttt{\textbackslash{}text} kann Text in Formeln eingefügt werden: $A_\text{optimal}$.



% Präzisierung der Aufgabenstelung
% *****************************************************************************

\section{Präzisierung der Aufgabenstellung}

Voraussetzung für die Präzisierung der Aufgabenstellung ist die Analyse des internationalen
Standes der Technik durch intensives Literatur- und Patentstudium zu
bereits vorhandenen Lösungen, offenen Fragestellungen und möglichen Lösungswegen.

Das Präzisieren der Aufgabenstellung erfolgt in Zusammenarbeit mit dem Betreuer und
unter Beachtung des aktuellen internationalen Standes auf dem Arbeitsgebiet. Folgende
Fragen sind zu klären:

\begin{itemize}
\item Ziel und Zweck der Arbeit,
\item Ausgangspunkt und Abgrenzung des Themas,
\item Art, Umfang, Form und Genauigkeit der zu erreichenden Arbeitsergebnisse,
\item Teilziele, Mittel und Methoden der Bearbeitung, Abhängigkeit der Teilergebnisse
voneinander.
\end{itemize}


% Hauptteil
% *****************************************************************************

\section{Erster Hauptabschnitt der Arbeit}
\subsection{Abschnitt innerhalb des Kapitels}
\subsubsection{Unterteilung innerhalb des Abschnitts}

Eine wissenschaftliche Arbeit zeichnet sich durch eine erkenntnis- und ergebnisorientierte Darstellung der Realisierung der einzelnen Teilaufgaben aus, was sie deutlich von einem Tätigkeitsbericht (Praktikumsbericht) unterscheidet. Beispielsweise könnte hier die Dimensionierung der Komponenten der Teilaufgabe 1 vorgestellt werden.\\\\
Hinweis zur Vorlage: Jeder Gliederungspunkt sollte mindestens eine Seite Text umfassen. Eine feinere Unterteilung der Überschriften als in diesem Beispiel sollte vermieden werden.
Absätze sind durch die Formatvorlage durch einen größeren Abstand voneinander getrennt, um eine Abgrenzung optisch zu unterstützen.



% Zusammenfassung
% *****************************************************************************

\section{Zusammenfassung}

Die abschließende Zusammenfassung hebt problemorientiert die Eigenschaften eines
Produktes, einer Lösungsmethode oder eines Programms hervor und zeigt Verbesserungen,
offene Probleme, Schlussfolgerungen und konkrete Vorschläge für weitere Forschungsarbeiten auf.

% Literaturverzeichnis
% *****************************************************************************

\clearpage
\phantomsection
\addcontentsline{toc}{section}{Literaturverzeichnis}
\begin{thebibliography}{999}	%damit es im inhaltsverz steht
\normalsize

\subsection*{Bücher}	%  * nimmt nummerierung raus
\bibitem {KR1} \textit{Krause, W.:} Gerätekonstruktion in Feinwerktechnik und Elektronik. 3.Aufl. 2011. [München, Wien]: Carl Hanser Verlag  - ISBN 3-446-19608-0


\subsection*{Datenblätter}	%  * nimmt nummerierung raus

\bibitem {SE05} LM1086.pdf Stand 2005: www.ti.com/lit/ds/symlink/lm1086.pdf\\ (27.05.2013)


\subsection*{Internet}	%  * nimmt nummerierung raus

\bibitem {St01} www.photoresearch.com/current/lrs455.asp (26.01.2013)


\end{thebibliography}

% Anhang
% *****************************************************************************
\clearpage
\appendix % in appendix.tex wurde der \appendix-Befehl verändert, so dass an seiner Stelle automatisch die Verzeichnisse des Anhangs eingefügt werden

\label{SecA}
\section{Messungen}

Achtung: Abbildungen und Tabellen im Anhang erscheinen \emph{nicht} im
Abbildungs- bzw.\@ Tabellenverzeichnis im Hauptteil der Arbeit, sondern in den
zugehörigen Verzeichnissen am Anfang des Anhangs!

In der Datei \texttt{appendix.tex} können die Befehle \texttt{\textbackslash
listofalofs} und \texttt{\textbackslash listofalots} auskommentiert werden, um
kein Abbildungs- bzw.\@ Tabellenverzeichnis für den Anhang zu generieren.

Zusätzlich kann die Definition des Befehls \texttt{\textbackslash
useappendixtocs} angepasst werden, so dass Abbildungen und Tabellen im Anhang
\emph{doch} in den Verzeichnissen im Hauptteil der Arbeit erscheinen (siehe
\texttt{appendix.tex}).


\label{SecA1}
\subsection{xxxx}
%\input{xxxx.tex}

%% Falls noch etwas nach dem Anhang kommt:
%\usestandardtocs
%\bookmarksetup{startatroot} % siehe bookmark-Anleitung
%\printindex % nur als Beispiel

\end{document}
