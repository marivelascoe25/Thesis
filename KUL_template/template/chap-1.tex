{\chapter{Background}}
\label{sec:background}

\section{Organic Semiconductors}

Semiconducting properties of conjugated polymers built by alterning electron donor and acceptor moieties \cite{matt_electronic_2021}, esta cita facil la voy a sacar, are nowadays attracted for applications where fast computing is not relevant, in bioapplications


\subsection{Electronic structure}

Since inorganic semiconductors' band theory does not take into consideration the Coulomb and exchange electron-electron interaction, which play a major role in organic semiconductors, it is necessary to add new theoretical approaches. On one hand, the transport properties are better described in terms of a hopping mechanism and the optoelectronic properties are better described by the molecular orbital picture. \cite{alcacer_electronic_2018}. Since the device under study in this work is a transistor and their transport properties in aqueous and quasi-solid environments, the theoretical approach used will be the hopping mechanism.

\subsection{Electronic transport: Hopping mechanism}

\subsection{Doping}
\subsubsection{Doping mechanism}
%\subsubsubsection{Molecular doping}
\paragraph{Molecular doping}
Use of small molecules
Electron-deficient dopants such as 2,3,5,6tetrafluoro-7,7,8,8-tetracyanoquinodimethane (F4TCNQ) extract electrons from shallow HOMO p-type OMIECs, increasing hole concentration \cite{tan_organic_2022}
%\subsubsubsection{Electrochemical doping}
\paragraph{Electrochemical doping}
Important in OECT physics covered in a later section.

\subsubsection{Measuring techniques to characterize doping}

\section{Organic Mixed Ionic/Electronic Conductors (OMIECs)}

Commonly semiconducting polymers which are redox-active and can simultaneously conduct ions and electrons. Electronic charges accumulated on the conjugated polymer backbone result in secondary property changes in electrochemical potential and electronic conductivity, allowing OMIECs to be implemented in a variety of devices such as chemical sensors, organic electrochemical transistors, and energy storage electrodes \cite{tan_organic_2022}

\subsection{A widely used material: PEDOT:PSS}

%\subsection{Engineering of semiconducting polymers}
\subsection{Other thiophene-based polymers}
%Seria chevere tener un grafico de como thiophene publications empezaron a crecer.
Thiophene is a planar conjugated ring structure consists of six delocalized pi-electrons. The aromatic nature arises from the four pi electrons and one unshared lone pair of electrons of the oxygen as six delocalized pi-electrons. It folow Hucke´s rule. Hene it is aromatic compound


\section{Organic Electrochemical Transistors (OECTs)}

devices that are mechanically compliant, biocompatible, and are sensitive to biochemical modules \cite{tan_organic_2022}

\subsection{Device physics}

\subsection{Operation modes}

\subsection{Important figures of merit}
\subsubsection{Transconductance}
\subsubsection{$\mu$C* product}
\subsubsection{Threshold voltage}

\subsection{Requirements to avoid undesirable side reactions}
Achieving effective charge transfer between the analyte and OMIEC requires appropriate alignment of the electrochemical potential of electrons on the OMIEC electrode and the redox specie. Failure to do so may result in the subsequent transfer of charges to other redox-active sinks in the environment, leading to undesirable side reactions and products that may interfere with the OMIEC’s operation. Electrons flow from a region of higher to lower electrochemical potential. Hence, achieving electron transfer from redox-active species to the OMIEC requires the latter to have a deep LUMO (high electron affinity) \cite{tan_tuning_2022}


\subsection{Building block for neuromorphic and bioelectronic applications}

%%% Local Variables: 
%%% mode: latex
%%% TeX-master: "thesis"
%%% End: 
