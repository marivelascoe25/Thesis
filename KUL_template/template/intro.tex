\chapter{Introduction}
\label{cha:intro}

The field of organic electronics has witnessed significant advancements in recent years due to its potential biocompatibility, mechanical compliant, and other application-specific characteristics. Among the numerous types of organic devices, Organic Electrochemical Transistors (OECTs) have attracted considerable attention due to their unique capabilities such as high transconductance and steep-slope switching at low operation voltages, which give them potential for use in energy storage, bioelectronics, and neuromorphic devices. 

Accumulation-mode transistors, devices that are normally in the OFF state at zero-gate-biased condition, rely on the use of undoped conjugated polymers. In contrast, the ability to precisely control and tune threshold voltage of an OECT can be achieved by manipulating the doping level of the mentioned conjugated polymer as shown by Tan et al. \cite{tanTuningOrganicElectrochemical2022}.  However, they fabricated devices that did not follow a complete microstructuring technique, limiting their integration into circuits. 

%\section{Motivation}
%Many challenges have raised for OECTs' operation, some of them include and will be the main focus of this thesis:
%\begin{itemize}
%\item The fabrication of accumulation-mode transistors, devices that are normally in the OFF state at zero-gate-biased condition.
%\item The modulation of their threshold voltage that allow the integration to different circuit designs.
%\end{itemize}

%\section{Goal}
%Understanding the relationship between the doping level and threshold voltage shift is crucial for the design and fabrication of high performance OECTs

The primary objective of this research is to address this missing information by developing a microstructuring method to fabricate accumulation-mode OECTs with controlled doping levels and enable their seamless integration into circuits. However, during the pursuit of this objective, it was identified that stability of the polymer and dopants in an electrochemical environment may pose challenges that need to be addressed as well. Therefore, this research project specifically aims to:

\begin{enumerate}
\item Characterize 3-(2-(2-(2-methoxyethoxy)ethoxy)ethoxy)thiophene (p(g3T2-T)) with varying doping levels of F$_{4}$TCNQ and F$_{6}$TCNNQ. This involves chemically modifying the conjugated polymer with different concentrations of dopant and analyzing their electronic structure, morphology and electrical properties using techniques such as UV-Vis spectroscopy, Ultraviolet Photoelectron Spectroscopy (UPS), %Atomic Force Microscopy (AFM), 
Van Der Pauw method, Electrical Impedance Spectroscopy (EIS) and Cyclic Voltammetry (CV),

\item fabricate OECT devices, which involves utilizing the conjugated polymer at different doping levels, and adapting an existing method that combines electrode patterning techniques, spin-coating, photolithography and inkjet printing. The devices will be carefully optimized to ensure reproducibility and stability,
 
\item assess doped polymer stability in OECT, which involves performing conductivity measurements over time and selecting an electrolyte composition that allows an stable performance, and
 
\item investigate the shift in threshold voltage through electrical characterization of the fabricated OECT devices with varying doping levels of F$_{4}$TCNQ and F$_{6}$TCNNQ.


\end{enumerate}

%\section{Structure of the Work}
The thesis is structured as follows: Chapter 1 provides an overview of organic electronics and the importance of the OECT threshold voltage, and outlines the motivation, goals, and structure of the thesis. Chapter 2 presents a comprehensive review of the relevant background information on Organic Semiconductors (OSCs), Organic Mixed Ionic Electronic Conductors (OMIECs), and Organic Electrochemical Transistors, and relevant research on tuning the threshold voltage of OECTs. Chapter 3 illustrates the Experimental Methods used in this research, describes the materials, equipment, software, and procedures to chemically dope the films and characterization method employed in the study. Finally, it outlines the fabrication and characterization process of OECT devices. In Chapter 4, the experimental results obtained are presented, from the characterization of the conjugated polymer at different doping levels and fabricated OECT devices. Analysis of the relationship between the doping level and the shift in threshold voltage. Finally, it discusses the implications of the findings and their relevance to the field. Lastly, Chapter 5 provides a summary of the research objectives and the extent to which they were achieved, suggests future research direction, and potential applications.

%By addressing the aforementioned research goals and following the proposed thesis structure, this study aims to contribute in adapting an existing protocol to the conjugated material p(g3T2-T), in understanding the doping-dependent tuning of OECT threshold voltage and establish a foundation for the development of high-performance organic electronic devices.
This study aims to adapt an existing protocol for the conjugated material p(g3T2-T), understand doping-dependent tuning of OECT threshold voltage, and establish a foundation for high-performance organic electronic devices by addressing research goals and following the proposed thesis structure.

%%% Local Variables: 
%%% mode: latex
%%% TeX-master: "thesis"
%%% End: 
