\chapter{Introduction}
\label{cha:intro}

The field of organic electronics has witnessed significant advancements in recent years due to its biocompatibility, mechanical compliant, and other application-specific characteristics. Among the numerous types of organic devices, Organic Electrochemical Transistor (OECT) has attracted considerable attention due to its unique capabilities such as high transconductance and steep-slope switching at low operation voltages, which give them potential for use in energy storage, bioelectronics and neuromorphic devices. 

%\section{Motivation}
Many challenges have raised for OECTs' operation, some of them include and will be the main focus of this thesis:
\begin{itemize}
\item The fabrication of accumulation-mode transistors, devices that are normally in the OFF state at zero-gate biased.
\item The modulation of their threshold voltage that allow the integration to different circuit designs.
\end{itemize}

While accumulation-mode transistors relies in the use of undoped conjugated polymers, the ability to precisely control and tune threshold voltage of an OECT can be achieve by manipulating the doping level of the mentioned conjugated polymer \cite{tan_tuning_2022}. Understanding the relationship between the doping level and threshold voltage shift is crucial for the design and fabrication of high performance OECTs, and it is the primary objective of this research.
%\section{Goal}

Specifically, it is aim to:

\begin{enumerate}
\item Characterize 3-(2-(2-(2-methoxyethoxy)ethoxy)ethoxy)thiophene (p(g3T2-T)) with varying doping levels. This involves chemically modifying the conjugated polymer with different concentrations of dopant and analyzing their electronic structure, morphology and electrical properties using techniques such as UV-Vis spectroscopy, Ultraviolet Photoelectron Spectroscopy (UPS), Atomic Force Microscopy (AFM), Van Der Pauw method, Electrical Impedance Spectroscopy (EIS) and Cyclic Voltammetry (CV).

\item Fabricate OECT devices, which involves utilizing the conjugated polymer at different doping levels, and adapting an existing method that combines electrode patterning techniques, spin-coating, photolithography and inkjet printing. The devices will be carefully optimized to ensure reproducibility and stability.
 
\item Investigate the shift in threshold voltage through electrical characterization of the fabricated OECT devices with varying doping levels of p(g3T2-T).

%\item Assess device stability: to evaluate stability of the OET devices, we will perform long-term measurements and examine the changes in their threshold voltage over time. This analysis will provide insights into the durability and reliability of the devices, essential for practical applications.

\end{enumerate}

%\section{Structure of the Work}
The thesis is structure as follows Chapter 1 provides an overview of organic electronics and the importance of the OECT threshold voltage. Outlines the motivation, goals, and structure of the thesis. Chapter 2 presents a comprehensive review of relevant background information on Organic Semiconductors (OSCs), Organic Mixed Ionic Electronic Conductors (OMIECs), and Organic Electrochemical Transistors (OECTs). And relevant research on tuning the threshold voltage of OECTs. Following to Chapter 3, it is illustrated the Experimental Methods used in this research, describes the materials, equipment, software, and procedures to chemically doped the films and characterization method employed in the study. Finally, outlines the fabrication and characterization process of OECT devices. In Chapter 4, the experimental results obtained are presented, from the characterization of the conjugated polymer at different doping levels and fabricated OECT devices. Analysis of the relationship between the doping level and the shift in threshold voltage. Finally, discusses the implications of the findings and their relevance to the field. Lastly, the Chapter 5, provides a summary of the research objectives and the extent to which they were achieved, suggests future research direction and potential applications.

By addressing the aforementioned research goals and following the proposed thesis structure, this study aims to contribute in adapting an existing protocol to the conjugated material p(g3T2-T), in understanding the doping-dependent tuning of OECT threshold voltage and establish a foundation for the development of high-performance organic electronics devices.

%%% Local Variables: 
%%% mode: latex
%%% TeX-master: "thesis"
%%% End: 
