\chapter{Results and Discussion}
\label{cha:3}

\section{Doping characterization}

%As followed in the procedure part...


%\begin{figure}
%  \centering
%  \includegraphics[width=10cm]{Images/dopingprocess.png}
%  \caption[Energy diagram of p(g3T2-T) and dopants F$_{6}$TCNNQ and F$_{4}$TCNNQ]{Energy diagram of p(g3T2-T) with a) F$_{6}$TCNNQ and b) F$_{4}$TCNNQ. Ionization potential of p(g3T2-T) extracted from reference \cite{tanTuningOrganicElectrochemical2022} and electron affinities of the neutral species (EA$^{0}$) and of the anion (EA$^{-}$) of the dopants are extracted from cyclic voltammetry measurements in reference \cite{kieferDoubleDopingConjugated2019}.}
%  \label{fig:doping}
%\end{figure}


\subsection{Absorbance and air stability}

\subsection{Workfunction shift}

\subsection{Thickness, sheet resistance and resistivity}

%\subsection{Roughness}

\section{Fabrication of Organic Electrochemical Transistors}
%Prior biasing gate, which is due to passive (ion) diffusion?

%The fact that solution-sequential doped films allows better homogeneity \cite{jacobsComparisonSolutionmixedSequentially2016}, make it also more compatible with microstructuring processes such as photolithography. At the expense of enabling easier and more precise control of doping levels \cite{tanOrganicMixedIonic2022}.

%The Ag/AgCl gate electrode’s work function is reasonably constant, the work function of an OMIEC gate electrode however may vary depending on its processing history and redox reactions with other species present in the electrolyte (e.g. molecular oxygen).28 Applying VGS only determines the potential difference between the gate and channel but does not control the potentials of either electrode (hence the position of the Fermi level) with respect to a reference. This leads to many challenges in operating an OECT with OMIEC gate electrodes.


%Achieving effective charge transfer between the analyte and OMIEC requires appropriate alignment of the electrochemical potential of electrons on the OMIEC electrode and the redox specie. Failure to do so may result in the subsequent transfer of charges to other redox-active sinks in the environment, leading to undesirable side reactions and products that may interfere with the OMIEC’s operation. Electrons flow from a region of higher to lower electrochemical potential. Hence, achieving electron transfer from redox-active species to the OMIEC requires the latter to have a deep LUMO (high electron affinity) \cite{tanMixedIonicElectronic2022} %paper

\subsection{Influence of doping OECT channel}

\subsection{Stability on air of p(g3T2-T)}

\subsection{Reverse oxidation of undoped-p(g3T2-T)}

\subsubsection{By electrochemical dedoping}

\subsubsection{By heating}


\subsection{Solid-OECTs using undoped-p(g3T2-T)}

\subsubsection{OECT with dropcast Solid-State Electrolyte}

\subsubsection{OECT with photopatternable Solid-State Electrolyte}

\subsubsection{OECT with inkjet-Printed Solid-State Electrolyte}

\subsection{Solid-OECTs with doped-p(g3T2-T)}



%\section{Conclusion}
%\lipsum[86-88]

%%% Local Variables: 
%%% mode: latex
%%% TeX-master: "thesis"
%%% End: 
