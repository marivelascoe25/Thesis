\chapter{Conclusions and Outlook}
\label{cha:conclusion}
%The final chapter contains the overall conclusion. It also contains suggestions for future work and industrial applications.

Besides the challenges of working with the conjugated polymer p(g3T2-T), due to its lower HOMO level that make it susceptible to molecular oxygen in ambient conditions and its highly volatile solvent, it was possible to successfully fabricate micro-structured accumulation-mode OECTs and solid-OECTs in ambient conditions. The section devoted to Doping characterization in the Results and Discussions Chapter demonstrated that it was possible to increase doping levels by using solution sequential doping. The level of homogeneity required to allow photopatterning processes was also achieved. However, accurate control of the doping level remains a challenge, an estimation of the molecular percentage of dopants cannot be assure due to the unknown quantity of dopants that succesfully diffuse throughout the thickness of the polymer, the characterization techniques used in this work present limitations to perform a full scanning of the doping level of the material within all its volume. Certainly, there is room for improvement in trying to achieve the level of control required. 



%%% Local Variables: 
%%% mode: latex
%%% TeX-master: "thesis"
%%% End: 
