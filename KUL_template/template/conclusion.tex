\chapter{Summary and Outlook}
\label{cha:conclusion}
%The final chapter contains the overall conclusion. It also contains suggestions for future work and industrial applications.

\section{Summary and conclusions}

During the development of this thesis, it was reviewed the increase potential of organic electrochemical transistors (OECTs) in bioelectronics applications. The background chapter was dedicated to address the fundamental concepts of OECTs, important figures of merit and side reactions specially during operation in air-saturated environments. Attention has been given to the fabrication of accumulation-mode devices with turn-on voltages close to zero with the aim to minimize power consumption. The use of dopants to tune the threshold voltage of OECTs was also tested and discussed in this work in a device architecture that is fully microstructured and hence will allow circuit integration. 

As a first step, it was important to define a processing technique that would allow not only different levels of doping but sufficiently homogeneous films that are compatible with photolithography. Different doping levels were achieved by dynamic spin coating p(g3T2-T) with different F$_{4}$TCNQ concentrations, evidenced by i) an increase in conductivity, ii) the formation of polarons as seen in the absorption spectra, and iii) an increase in the workfunction seen in the kinetic/binding energy spectra of photolectrons.

Subsequently, devices with a polarizable gate (Ag/AgCl) and, p(g3T2-T) channel at different doping levels coupled with a solid-state electrolyte (SSE) precursor was fabricated and study under normal cleanroom conditions. Transconductance had a negative impact and unexpectedly, threshold voltage shifted towards negative values as dopant concentration increased. Suggesting a compensation doping and/or TCNQ$^{-}$ anions drawing away from the polymer since they are not chemically bind with p(g3T2-T). Meanwhile, pristine p(g3T2-T) channel oxidized upon interaction with oxygen in the environment during the photolithography process leading to a normally ON device which was evidenced by monitoring the channel conductivity before and after patterning.

It was important to understand the stability of devices upon contact with the SSE precursor. Therefore, channel conductivity was monitored. Interestingly, a spontaneous decay of almost eight orders of magnitude was perceived for both pristine (oxidized) and doped p(g3T2-T) with F$_{6}$TCNNQ under nitrogen environment (dedoping), and less than two orders of magnitude decay in environmental conditions, suggesting simultaneous dedoping and ORR. 

A prior electrochemical dedoping step was then implemented to obtain devices with negative threshold voltage. Additionally, cyclic voltammetry (CV) was performed with Ag/AgCl gate to understand important redox points where Oxygen Reduction Reaction (ORR) may take place and the possibility of degradation due to the formation of hydrogen peroxide. It was found that huge electron transfer in values below -0.5 V (outside the turn on voltage), Hidalgo et al. attributed a similar peak value to ORR \cite{hidalgocastilloSimultaneousPerformanceStability2022a}.

The fabrication of side-gate OECTs using a microstructure pristine p(g3T2-T) gate coupled with a dropcasted and microstructured SSE, the latter achieved by photolithography and ink-jet printing, was performed. Microstructured SSE devices were washed with ethanol and applied a baking step to add a promoter for assuring adhesion with SSE but additionally it aimed in counteracting oxidation of polymer, so prior electrochemical de-doping was not consider necessary. Transfer characteristics of devices evidenced negative threshold voltages (with exception of the ink-jet printing), and interestingly exhibiting redox peaks outside the turn-on voltages values under nitrogen environment, that were many times the most prominent. We hypothesized that these peaks correspond to ORR caused by the remained oxygen in the polymer, but other side reactions between SSE components may occur. 

Ink-jet printed SSE did not achieve negative values due to its major exposure time under oxygen during the printing process. Photolithography seems to be most compatible technique to achieved fully microstructure side-gate OECTs with.

Finally, the fabrication of a microstructure side-gate OECT using doped p(g3T2-T) with 5 mg/mL concentration of F$_{6}$TCNNQ was held with ink-jet printed SSE. Minor threshold voltage shift towards positive values was achieved, the ORR and spontaneous de-doping during the printing process was hypothesized to be the cause, hence, overshadowing the doping effect. But guarantying a better stability and enhancing performance (higher transconductance, channel and volumetric capacitance).

\section{Outlook}

Some of the major problems during the fabrication process, was the low yields achieved even during the patterning of the material. The success of dynamic spin coating homogeneous films relies much in minimizing human error. This technique was chosen due to the volatility of chloroform solvent, finding an alternative solvent that could also ensure polymer concentration and allow the possibility of printing, desired also for better control of film thicknesses would give more reliable calculations on channel and volumetric capacitance.

Oxidation of pristine p(g3T2-T) is unavoidable unless working entirely under nitrogen atmosphere. Therefore, proper treatment to counteract oxidation needs to be addressed. Optimization of the ethanol rinsing and baking step can be study since this approach does not add an extra step in the fabrication process. Addressing this issue may lead us to improve yields on operational devices. Although many devices were not operated under environmental conditions, degradation was still perceived.

When obtaining a pristine p(g3T2-T) operation under environmental conditions will still undergo ORR unless encapsulated. Another approach is the incorporation of additives such as BCF, a strong Lewis acid, as demonstrated by Hidalgo Castillo \cite{hidalgocastilloSimultaneousPerformanceStability2022a} that will serve as sacrificial molecule to prevent p(g3T2-T) from oxygen interaction and unavoidable degradation. 

Stability of doping level of the polymer upon contact with SSE needs also further investigation, while it was found that after SSE crosslinking (gel formation) the stability of the channel conductivity drops significantly less than upon contact with SSE precursor. It was discussed that photopattern SSE may be an approach to obtain microstructured side-gate OECT without overshadowing \textbf{much} the doping effect in the device. However, other approaches can be evaluated, but it is important to first investigate if compensation doping is happening or anions are leaving our polymer side chains or both. This can be determined by performing tests with spectroelectrochemistry. An alkaline adhesion promoter before SSE may prevent both effects happen spontaneously, and other approaches such as creating binding site for dopants. 


%However, stability might be a concern where intentional doping might help

%Add hypothesis for low yield -> side reactions?


%%% Local Variables: 
%%% mode: latex
%%% TeX-master: "thesis"
%%% End: 
