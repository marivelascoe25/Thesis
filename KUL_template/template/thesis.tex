\documentclass[master=eelt,masteroption=ec]{kulemt}
\setup{% Remove the "%" on the next line when using UTF-8 character encoding
  inputenc=utf8,
  title={Tuning of Organic Electrochemical Transistor Threshold Voltage by varying doping of semiconducting polymer p(g3T2-T)},
  author={Marielena Velasco Enriquez},
  promotor={Prof.\,Dr.\,Karl Leo\and Prof. Dr. Steven De Feyter},
  assessor={PD Dr.rer.nat.habil. Hans Kleeman},
  assistant={Anton Weissbach}}
% Remove the "%" on the next line for generating the cover page
%\setup{coverpageonly}
% Remove the "%" before the next "\setup" to generate only the first pages
% (e.g., if you are a Word user).
%\setup{frontpagesonly}

% Choose the main text font (e.g., Latin Modern)
\setup{font=lm}

% If you want to include other LaTeX packages, do it here. 

% Finally the hyperref package is used for pdf files.
% This can be commented out for printed versions.
%\usepackage[pdfusetitle,colorlinks,plainpages=false]{hyperref}
\usepackage[pdfusetitle,hidelinks,plainpages=false]{hyperref}

%%%%%%%
% The lipsum package is used to generate random text.
% You never need this in a real master's thesis text!
\IfFileExists{lipsum.sty}%
 {\usepackage{lipsum}}%
 {\newcommand{\lipsum}[1][1-7]{\par And some text: lipsum ##1.\par}}
%%%%%%%

%\includeonly{chap-n}
\begin{document}

\begin{preface}
  I would like to thank everybody who kept me busy the last year,
  especially my promoter and my assistants. I would also like to thank the
  jury for reading the text. My sincere gratitude also goes to my wive and
  the rest of my family.
\end{preface}

\tableofcontents*

\begin{abstract}
Organic Electrochemical transistors (OECTs) exhibit advantageous properties, such as high transconductance and steep-slope switching, while operating at very low voltages. Although, their switching speed is comparatively slower than solid-state devices, it remains sufficient for applications in bioelectronics \cite{rivnay_organic_2018}. The gold standard for p-type OECT devices is PEDOT:PSS. However, its main drawback lies in its depletion-mode operation, which requires power to turn off the device. To minimize power consumption and improve stability, efforts have been made to the design semiconducting polymers that allow accumulation-mode devices. One such polymer, 3-(2-(2-(2-methoxyethoxy)ethoxy)ethoxy)thiophene (p(g3T2-T)) has demonstrated threshold voltages close to zero and high transconductance \cite{nielsen_molecular_2016}. Furthermore, by doping p(g3T2-T) at various levels and drop-casting it as a gate, it has been possible to fine-tune the threshold voltage \cite{tan_tuning_2022}. This study aims to adapt a microstructuring method for fabricating OECT devices that incorporate a solid electrolyte \cite{weissbach_photopatternable_2022} along with different doping levels of p(g3T2-T). Additionally, the study aims to adjust the threshold voltage by utilizing these varying doping levels.

  %\lipsum[1]
\end{abstract}

% A list of figures and tables is optional
%\listoffigures
%\listoftables
% If you only have a few figures and tables you can use the following instead
\listoffiguresandtables
% The list of symbols is also optional.
% This list must be created manually, e.g., as follows:
\chapter{List of Abbreviations and Symbols}
\section*{Abbreviations}
\begin{flushleft}
  \renewcommand{\arraystretch}{1.1}
  \begin{tabularx}{\textwidth}{@{}p{12mm}X@{}}
    LoG   & Laplacian-of-Gaussian \\
    MSE   & Mean Square error \\
    PSNR  & Peak Signal-to-Noise ratio \\
  \end{tabularx}
\end{flushleft}
\section*{Symbols}
\begin{flushleft}
  \renewcommand{\arraystretch}{1.1}
  \begin{tabularx}{\textwidth}{@{}p{12mm}X@{}}
    42    & ``The Answer to the Ultimate Question of Life, the Universe,
            and Everything'' according to \cite{h2g2} \\
    $c$   & Speed of light \\
    $E$   & Energy \\
    $m$   & Mass \\
    $\pi$ & The number pi \\
  \end{tabularx}
\end{flushleft}

% Now comes the main text
\mainmatter

\chapter{Introduction}
\label{cha:intro}
The first contains a general introduction to the work. The goals are
defined and the modus operandi is explained.
The chapter should broadly contextualize your research and motivate your work.

%\section{Motivation}
%Here you motivate, why you are doing your research.

%\section{Goal}
%This section should clarify, what should be achieved by the work.

%\section{Structure of the Work}
%Here the structure can be \emph{briefly} explained.


%%% Local Variables: 
%%% mode: latex
%%% TeX-master: "thesis"
%%% End: 

{\chapter{Background}}
\label{sec:background}

\section{Organic Semiconductors}

Semiconducting properties of conjugated polymers built by alternating electron donor and acceptor moieties \cite{mattinenAtomicLayerDeposition2021}, 


\subsection{Electronic Structure}

Since inorganic semiconductors' band theory does not take into consideration the Coulomb and exchange electron-electron interaction, which play a major role in organic semiconductors, it is necessary to add new theoretical approaches. On one hand, the transport properties are better described in terms of a hopping mechanism and the optoelectronic properties are better described by the molecular orbital picture \cite{alcacerElectronicStructureOrganic2018}. Since the device under study in this work is a transistor and their transport properties in aqueous and quasi-solid environments, the theoretical approach used will be the hopping mechanism.

\subsection{Molecular Doping}

Use of small molecules
Electron-deficient dopants such as 2,3,5,6 tetrafluoro-7,7,8,8-tetracyanoquinodimethane (F$_{4}$TCNQ) extract electrons from shallow HOMO p-type OMIECs, increasing hole concentration \cite{tanOrganicMixedIonic2022}.

\subsubsection{Measuring techniques to characterize doping}

\section{Organic Mixed Ionic/Electronic Conductors (OMIECs)}

Initially investigated for batteries and super capacitors, where the induction of charges was the main objective of a semiconducting material, OMIECs has rapidly grown to include other applications, among them: OECTs \cite{paulsenOrganicMixedIonic2020}.

% PEDOT:PSS Heterogeneous, blends or complexed systems OMIEC. Conjugated polymer/electrolyte blends. Contains ions chemically linked ot an insulating or conjugated component
% p(g3T2-T) Homogeneous, single-component systems OMIEC. Conjugated polymer electrolytes. Ions introduced as free species upon material casting or device operation

Commonly semiconducting polymers which are redox-active and can simultaneously conduct ions and electrons. Electronic charges accumulated on the conjugated polymer backbone result in secondary property changes in electrochemical potential and electronic conductivity, allowing OMIECs to be implemented in a variety of devices such as chemical sensors, organic electrochemical transistors, and energy storage electrodes \cite{tanOrganicMixedIonic2022}.

\subsection{A widely used material: PEDOT:PSS}
%\subsection{Engineering of semiconducting polymers}
Due to its commercial availability, operational stability, and relatively high performance, the conductive polymer poly(3,4-ethylenedioxythiophene) poly(styrene-sulfonate) (PEDOT:PSS) became a standard material for p-type OECTs. Its main drawback lays in its depletion-mode operation, which requires a voltage to turn off the device (as represented in Figure 1(b)). With the aim of minimizing power consumption, there is a special interest to design semiconducting polymers that would allow accumulation-mode devices (Figure 1(c)) with high performance \cite{nielsenMolecularDesignSemiconducting2016} \cite{tanOrganicMixedIonic2022}.

\subsection{Other Thiophene-based polymers}
%Seria chevere tener un grafico de como thiophene publications empezaron a crecer.
Thiophene is a planar conjugated ring structure consists of six delocalized pi-electrons. The aromatic nature arises from the four pi electrons and one unshared lone pair of electrons of the oxygen as six delocalized pi-electrons. It folow Hucke´s rule. Hene it is aromatic compound

Nielsen et al. reported a series of semiconducting polymers with Ethylene Glycol (EG) side chains designed to elucidate important strcuture-property guidelines for accumulation-mode OECT operation. They demonstrated that an OECT with 3-(2-(2-(2-methoxyethoxy)ethoxy)ethoxy)thiophene p(g3T2-T), as seen in Figure , has higher transconductance, and a turn-on voltage close to zero compared to other thiophene-based species \cite{nielsenMolecularDesignSemiconducting2016}. While its backbone design warrant reversibility during electrochemical redox reactions and good electronic transport, the EG side chains enable its stability in aqueous electrolytes and efficient transport of ionic and electronic charge carrier \cite{moiaDesignEvaluationConjugated2019}. Moser et al. studied the impact of the length of the EG side chain of this polythiophene backbone on the performance of OECTs. They reported that reducing chain length would maximize the capacitance, but the increase of length would enhance ion-polymer interaction. Finally, they suggested an optimum length of 3 monomers in side chains over 2, 4 and 6 monomers \cite{moserEthyleneGlycolBasedSide2020}. The main advantage of p(g3T2-T) over PEDOT:PSS is its higher transconductance \cite{nielsenMolecularDesignSemiconducting2016}, the absence of extensive pre- and postprocessing to optimize polymer stability and electrochemical performance in aqueous media and the possibility of an accumulation-mode OECT with low operation voltages\cite{moserEthyleneGlycolBasedSide2020}.

Organic semiconductors with polar sidechains have been identified as a promising class of materials for the field of bioelectronics. These materials, also called organic mixed ionic/electronic conductors (OMIECs), can exchange ions with aqueous electrolytes when electronic charge carriers are injected, transported, and stored in the bulk of the material \cite{giovannittiEnergeticControlRedoxActive2020}

Homogeneous single phase OMIECs (types V and VI) display larger magnitudes of ionic–electronic coupling and larger values of volumetric capacitance than biphasic OMIECs (types I–IV) \cite{paulsenOrganicMixedIonic2020}. 


%actually from reference 44, benchmarking omiecs for transistors

\subsection{Electrochemical Doping}

The electrochemical charging of OMIECs can be described as a capacitive faradaic charging process,
% there is current caused by charge transfer, other physical phenomenas such as desorption or adsorption can also lead to the aparition of current that is non faradaic reaction
meaning that the OMIEC undergoes a change in its oxidation state (p-doping in the language of physicists) through an electron transfer with the contact (current collector), while ions from the electrolyte penetrate inside the channel material to compensate the carge carrier on the polymer backbone electrostatically with no change in the inserted ion's oxidation state \cite{giovannittiEnergeticControlRedoxActive2020}  

Savva et al. study the influence of water on the performance of OECT, the water uptake of conjugated polymer films led to 10-13\% mass increase under non biased conditions. As the concentration of water decrease (NaCl$_{aq}$ 10 mM, 100mM, 1M, and 6M) ionic charging is faster regardless of the doping pulse, however the fastest ionic charging is achieved at NaCl$_{aq}$ 1 M. The injection/drift of ions is also affected by the ion-counterion attractive forces which delays the ion injection from the electrolyte, opposing their drift into the polymer (NaCl$_{aq}$ 6 M hinders the drift of anions) affecting the response time. \cite{savvaInfluenceWaterPerformance2019}

\section{Organic Electrochemical Transistors (OECTs)}

Organic Electrochemical Transistors (OECTs) consists of metallic source, drain and gate electrodes, an organic semiconductor channel (specifically an OMIEC as described in previous section) and an electrolyte that couples channel and gate \cite{rivnayOrganicElectrochemicalTransistors2018}

\begin{figure}[ht]
	\centering
	\subfloat[]{{\includegraphics[height=4.5cm]{Images/structure.png} }}
	\qquad
	\subfloat[]{{\includegraphics[height=8cm]{Images/depletion.png} }}
	\subfloat[]{{\includegraphics[height=8cm]{Images/accumulation.png} }}
	\caption{(a) Typical structure of an organic electrochemical transistor (OECT). (b) Transfer curve showing depletion-mode operation of an OECT with a conducting polymer channel. (c) Transfer curve showing accumulation-mode operation of an OECT with a semiconducting polymer channel. Images extracted from reference \cite{rivnayOrganicElectrochemicalTransistors2018}.}
	\label{fig:modes}
\end{figure}

The semiconductor

%devices that are mechanically compliant, biocompatible, and are sensitive to biochemical modules \cite{tanMixedIonicElectronic2022} 

\subsection{Device Physics}

\subsection{Operation Modes}

Accumulation-mode devices has the advantage of dissipating less static power when the device is not operated, due to low OFF current, which must be minimized as much as possible \cite{giovannittiEnergeticControlRedoxActive2020}

\subsection{Important Figures of Merit}
\subsubsection{Transconductance}
\subsubsection{$\mu$C* product}
\subsubsection{Threshold voltage}

An approach to shift the operating voltage range for PEDOT:PSS OECTs to lower the channel current, leading to reduced power consumption, is to tune the threshold voltage by de-doping PEDOT:PSS using commercially available amine-based molecular de-dopants \cite{Keene_enhmod_pedot}. Tan et al., on the other hand, explored a different approach, rather than modifying the doping level of the channel, they tuned the doping level of the gate to shift the threshold voltage. They used p(g3T2-T) and obtained a 400mV change with 60\% mol ratio of 2,3,5,6-Tetrafluoro-7,7,8,8-tetracyanoquinodimethane (F4TCNQ) dopant. The advantage over this approach is i) protecting the material from oxidation, since the Fermi level in brought towards the highest occupied molecular orbital (HOMO), and ii) no interference with the channel which helps to leave the transconductance unaffected \cite{tanTuningOrganicElectrochemical2022}.

\subsection{Requirements to Avoid Undesirable Side Reactions}


Achieving effective charge transfer between the analyte and OMIEC requires appropriate alignment of the electrochemical potential of electrons on the OMIEC electrode and the redox specie. Failure to do so may result in the subsequent transfer of charges to other redox-active sinks in the environment, leading to undesirable side reactions and products that may interfere with the OMIEC’s operation. Electrons flow from a region of higher to lower electrochemical potential. Hence, achieving electron transfer from redox-active species to the OMIEC requires the latter to have a deep LUMO (high electron affinity) \cite{tanMixedIonicElectronic2022} %paper

\subsubsection{Oxygen Reduction Reaction (ORR)}

With the aim of developing accumulation-mode OECTs, the engineering of new OMIECs were introduced as commented in previous sections. Normally, this polymer backbones have low ionization potential (IPs) which lead to another side effect issue that little attention has been paid: non capacitive faradaic rections in ambient: electron-transfer reaction from the OMIEC to molecular oxygen described as oxygen reduction reaction (ORR)

% Remember 
% Thermodynamically favored processes or reactions are those that involve both a decrease in the internal energy of the components (ΔH° < 0) and an increase in entropy of the components (ΔS° > 0). These processes are necessarily “thermodynamically favored” (ΔG° < 0) or negative. ΔG° = ΔH°-TΔS°
% Catalyst help to reduce the activation energy required for a reaction

The ORR yields either H$_{2}$O$_{2}$ or water (H$_{2}$O) as well as charging (oxidation) of the OMIEC that acts as the catalyst. The first shows a free energy difference that is endergonic for OMIECs with IPs \> 4.9eV and hence prevent the OMIEC from undergoing ORR that form H$_{2}$O$_{2}$. To prevent the ORR in ambient conditions, OMIECs based on donor-acceptor copolymer (Type III or IV) that have large IPs to shift
\cite{giovannittiEnergeticControlRedoxActive2020}
% Actually this material does not avoid this, we need to take into account this interaction.

\subsection{Building Block for neuromorphic and bioelectronic applications}



%%% Local Variables: 
%%% mode: latex
%%% TeX-master: "thesis"
%%% End: 

%\chapter{Experimental Methods} \label{cha:2}
%\lipsum[77]

\section{Materials}
%\lipsum[78]
All reactives were purchased from commercial suppliers and non further chemical modification or purification was done unless stated before.

\begin{itemize}
\item Chromium etchant: Standard, Sigma Aldrich
\item Gold etchant: Standard, Sigma Aldrich HHPAA (2-Hydroxy-4’-(2-hydroxyethoxy)-2-methylpropiophenone), 98$\%$, Sigma Aldrich 
\item Developer: AZ 726 MIF Developer, Merck performance Materials GmbH
\item EG: Ethylene glycol, $\geq$ 95$\%$, Sigma Aldrich
\item $[$EMIM$][$EtSO$_{4}]$: (1-Ethyl-3-methylimidazolium ethyl sulfate), $\geq$95$\%$, Sigma Aldrich 
\item MBBAm (N,N’-Methylenebisacrylamide), 99$\%$, Sigma Aldrich 
\item NIPAm: (N-Isopropylacrylamide), 97$\%$, Alfa Aesar 
\item Sacrificial Layer 1: Sacrifical Layer 1 (SL1), Orthogonal Inc. 
\item Orthogonal Photoresist for undoped species: NLOF 2020, commercial negative-tone photoresist, Microchemical.
\item Orthogonal Photoresist for doped species: OSCoR 4020 Photoresist, Orthogonal Inc.
\item Developer for SL1: Developer HF 7300, Orthogonal Inc.
\item Orthogonal Developer for OSCor 4020: Orthogonal Developer 103a, Orthogonal Inc.
\item Photoresist: AZ 1518 Photoresist, Merck Performance Materials GmbH \& Microchemical GmbH 
\item Orthogonal Stripper: Orthogonal Stripper 900, Orthogonal Inc. 
\item p(g3T2-T): 3-(2-(2-(2-methoxyethoxy)ethoxy)ethoxy)thiophene
\item Dopants: 1,3,4,5,7,8-hexafluorotetracyanonaphthoquinodimethane (F$_{6}$TCNNQ) and 1,3,4,5-tetrafluorotetracyanonaphthoquinodimethane (F$_{4}$TCNQ)
\item Silane A174 (3-(Trimethoxysilyl)propyl methacrylate), TCI

\end{itemize}

\section{Equipment}
\begin{itemize}
\item Baking: All baking steps were carried out on a Stuart SD160 digital hotplate (Stuart Equipment, UK). 
\item Electrical characterisation (ambient): Device characterisation under ambient conditions was performed on a Everbeing C-6 Probe Station (Everbeing Int’l Corp., Taiwan), connected to a Keithley 4200-SCS Semiconductor Characterisation System (Keithley Instruments, USA). 
\item Electrical characterisation (glovebox): Device characterisation was performed in a nitrogen-filled glovebox. Probing needles were connected to two Keithley 236 Source Measure Units (Keithley Instruments, USA). 
\item Cyclic voltammetry and Impedance measurements: Measurements were carried out by using a Metrohm Autolab PGSTAT302N potentiostat/galvanostat (Metrohm AG, Switzerland) in the range of 10$^{-1}$ to 10$^{5}$ Hz. 
\item Micrographs: Micrographs were taken on a Nikon Eclipse LV100ND microscope, equipped with a DS-Fi2 camera (Nikon, Japan). 
\item Photolithography: Photolithography was carried out on a SÜSS Microtec MJB4 maskaligner system (SÜSS Microtec AG, Germany). 
\item Photomasks: Photomasks were custom made by Compugraphics Jena in a 4-inch format (soda-lime glass covered with chromium) and held several mask designs (Compugraphics Jena GmbH, Germany). 
\item Plasma cleaning: O2-plasma cleaning was performed by using a Harrick PDC-002 plasma cleaner (Harrick Plasma, USA), connected to a Leybold Heraeus Combitron CM 330 Vacuum Gauge Controller (Leybold GmbH, Germany). 
\item Plasma etching: O2-plasma etching was performed by using a Diener electronic ATTO plasma cleaner (Diener electronic GmbH \& Co. KG, Germany). 
\item Profilometry: Profilometry was performed on a Veeco Dektak 150 surface profiler (Veeco Instruments Inc., USA).
\item Sheet resistance measurements: The sheet resistance was measured using a linear four-point probe system (Lucas four-point probe connector) connected to a multimeter (Keithley 2010 Multimeter).
\item Absorption measurements: Measurements were performed with UV-Visible-NearInfraRed Spectroscopy on a SolidSpec-3700 UV-Vis-NIR spectrometer from Shimadzu.
\item Workfunction measurements: Worfunction was measured using Ultraviolet Photoelectron Spectroscopy (UPS) by using a PHOIBOS 100 from Specs, a Helium plasma discharge lamp (UVS10/35, Specs) for UV excitation line of 21.22 eV.
\item Spincoating: Samples were coated with a SAWATEC SM-180-BT spincoater (SAWATEC AG, Switzerland).
\end{itemize}

\section{Software} \label{param}
%% Add citations of libraries and stuff
\begin{itemize}
\item Data processing: All data was processed by customised scripts written in the Python programming language. Files import and manipulation was done by using OS \cite{os_module} and CSV \cite{csv_module} modules. Mathematical computations (e.g. fits, integration) were carried out by employing the NumPy \cite{numpy_2012}, SciPy \cite{scipy_linreg}, and PeakDetect \cite{peakdetect} libraries. Visualisations were performed using the Matplotlib library \cite{matplotlib_2012}. All is compiled in the following GitHub Repository: \href{https://github.com/marivelascoe25/Thesis.git}{\textbf{marivelasco25/Thesis.git}}.
\item Electrical characterisation: Measurements were performed by controlling SMUs through the in-house developed SweepMe! software \href{https://sweep-me.net/}{\textbf{(https://sweep-me.net/)}}. 
\item Profilometry: Profilometry was performed by using the Dektak software (Veeco Instruments Inc., USA).
\item Cyclic voltammetry: Measurements were performed by using the NOVA software (Metrohm AG, Switzerland). Parameters were fixed and are described in the following table: 

\begin{table}[h]
	\centering
	\caption{Cyclic Voltammetry parameters.}
	\begin{tabular}{r r} \hline
		Parameter	& Value \\ \hline
		Start potential	& 0 V \\ 
		Upper vertex potential	& 1 V \\ 
		Lower vertex potential	& -1 V \\ 
		Stop potential	& 0 V \\ 
		Number of scans	& 10 \\ 
		Scan rate	& 0.1 V/s \\ 
		Step	& 0.05 \\ \hline
	\end{tabular}
	\label{tab:CV}
\end{table}

\item Impedance measurements: Measurements were performed by using the NOVA software (Metrohm AG, Switzerland). Parameters were fixed and are described in the following table: 

\begin{table}[h]
	\centering
	\caption{Electrochemical Impedance Spectroscopy parameters.}
	\begin{tabular}{r r} \hline
		Parameter	& Value \\ \hline
		Initial frequency	& 10$^{5}$ Hz  \\ 
		Final frequency	& 10$^{-1}$ Hz \\ 
		Frequencies per decade	& 10 \\ 
		Amplitude	& 0.01$\cdot$V$_{RMS}$ \\ \hline
	\end{tabular}
	\label{tab:EIS}
\end{table}

%105 - 10-1
%10 frequencies per decade
%Amplitude 0.01 VRMS

\end{itemize}

%\section{Photomasks}
%Photomasks were employed during photolithography to cover or uncover the respective areas of interest, depending on the photoresist employed (positive or negative, respectively). Details are given in Chapter . The employed photomasks for OECT fabrication are shown in Figure , with a labeled close-up scheme of a transistor device given in Figure . All devices were designed in a side-gate structure with the 16 devices each mask comprised differing in channel length and gate distance. An overview of the specific dimensions is provided in Table  with the designations assigning the devices (U1 – U8 (Up) and D1 – D8 (Down)). Since investigations on device dimensions were not a direct study object of this work, according assignments were left out. However, all comparisons between different substrates do of course always refer to the very same devices on each individual sample. A device assignment of all plots shown is given in Table S1. It shall be pointed out that the fabrication process did regularly lead to samples with several non-functioning devices. Accordingly, experiments were to be conducted on OECTs that were found working, which accounts for the variation in examined devices. For experiments of Chapter , available photomasks were arranged and covered to yield the setup shown in Figure . For experiments of Chapter , channels of p(g3T2-T) with underlying gold contacts were prepared. The corresponding photomasks are shown in Figure  and were printed on plastic foil in a common inkjet printer. As schematically shown in Figure  though, the lower big as well as several small gold contacts have not been used during execution. The lowest small gold contact served as source electrode.

\section{Experimental Procedures}

All fabrication steps were performed under standard cleanroom conditions.
\subsection{Preparation of films}
\paragraph{Dynamic spin-coating of p(g3T2-T).}Unlike reference \cite{tanTuningOrganicElectrochemical2022}, in which drop cast is used to deposit undoped and doped p(g3T2-T), in order to perform photolithography and being able to do a miniaturization process, homogeneous films are needed. Dynamic spin-coating was previously stablished by BioSens group members at IAPP, due to the high volatility of p(g3T2-T) solvant: chloroform. Substrates were cleaned using subsequent steps of ultrasonic bath with acetone for 15 minutes, IPA rinsing, N$_{2}$ drying and O$_{2}$-plasma cleaning. Then, 70$\mu$L of 10 mg/mL of p(g3T2-T) mixed at 60$^{\circ}$C for 20 minutes, was dynamic-spin-coated at 3000 RPM for 60s, to yield approximately 70nm-thick films. Dry sample at 80$^{\circ}$C.

\paragraph{Dynamic spin-coating of F$_{4}$TCNQ dopant.}Different doping levels of p(g3T2-T) was achieved by dynamic spin-coating 140$\mu$L of dopants at different concentrations (5, 10 and 20 mg/mL), previously mixed in acetronitrile at $60^{\circ}$C for 20 minutes. Dry sample at 80$^{\circ}$C.

\begin{figure}[ht]
  \centering
  \includegraphics[width=10cm]{Images/pdf/spin_coating.pdf}
  \caption{Dynamic spin coating process to obtain undoped and doped films of p(g3T2-T)).}
  \label{fig:coating}
\end{figure}

\subsection{Doping characterization of films}

\paragraph{UV-Vis-NIR Spectroscopy.}After the preparation of the films, Transmittance (T) and reflectance (R) was measured using the UV-Vis-NIR Spectrometer in the range of 285 to 1600 nm, and slit width of 5,0 nm. Then absorption (A) was calculated via


\begin{equation}\label{eq:abs}
	A = 1 - T - R,
\end{equation}

normalized with respect to the incoming light. Additional measurements were taken, after some days of storage under ambient conditions to check its stability on air.

\paragraph{Ultraviolet Photoelectron Spectroscopy.}After the preparation of the films (air exposure) the energy of the highest occupied molecular orbital cutoff (E$_{HOMO}$) and the high binding energy cutoff E$_{HBEC}$ was measuring using UPS. The pressure in the chamber during measurements was about 5$\cdot$10$^{-9}$ mbar, while the base pressure is in the range of 10$\cdot$10$^{-10}$ mbar. The workfunction (WF) and ionization energy (IE) are given by the following equations:

\begin{equation}\label{eq:wf}
	WF = h\nu - E_{HBEC},
\end{equation}

\begin{equation}\label{eq:ie}
	IE = h\nu - (E_{HBEC}-E_{HOMO}),
\end{equation}

where $h\nu = 21.22 eV$, the main He I excitation line \cite{buchholtzDopingPropertiesNovel2021}. 

\paragraph{Profilometer.}The films were scratched 4 times to remove part of material, and measure the cavity dept to obtain the film thickness.

\paragraph{Van Der Pauw.}Sheet resistance was calculated via

\begin{equation}\label{eq:rs}
	R_{S} = \frac{\pi}{ln(2)}R = 4,53 R \, [\Omega/sq],
\end{equation}

where R is the resistance measured with the four-point probe setup and resistivity was calculated using the following equation:

\begin{equation}\label{eq:resist}
	\rho = R_{s}t \, [\Omega.cm],
\end{equation}
 
%\subsection{Roughness characterization of films} 

\subsection{Fabrication of Organic Electrochemical Transistors}

\subsubsection{Influence of doping OECT channel} \label{subsec:photo}

Photolitographically patterned substrates with gold contacts for source, drain and gate were obtained following references \cite{weissbachPhotopatternableSolidElectrolyte2022} \cite{bongartzOrganicElectrochemicalTransistors2021}, described in Figure \ref{fig:aupat}.

\begin{figure}[ht]
	\centering
	\includegraphics[width=10cm]{Images/pdf/Au-patterning.pdf}
	\caption[Visualization of the workflow for Au-contacts patterning.]{Visualization of the workflow for Au-contacts patterning. Image modified from reference \cite{bongartzOrganicElectrochemicalTransistors2021}.}
	\label{fig:aupat}
\end{figure}

Then p(g3T2-T) films were deposited following the procedure described in previous section. The patterning of channel follow a similar procedure from the same reference where PEDOT:PSS is being used, with some modifications in the exposure and developing times. However, if the undoped-p(g3T2-T) is being used, the protocol needs the addition of a sacrificial layer to ensure cross-linking. Both processes are described in the following:

\paragraph{Patterning undoped p(g3T2-T)}Patterning of the p(g3T2-T) was achieved with photolithography, as shown in Figure \ref{fig:undopedpat}, first a fluoropolymer Sacrificial Layer 1 was spin coated at 6000 RPM for 60 s, followed by a baking step for 60 s at $113^{\circ}$C. Before the deposition of photoresist, a 60 s-plasma treatment was applied from UFO2 to promote adhesion. Then, NLOF 2020 was spin coated at 3000 RPM for 60 s and baked for another 60 s at $113^{\circ}$C. Exposure of negative resist was performed for 12 s with shadowing all areas except the ones of interest. After post-baking for 60 s at $113^{\circ}$C, NLOF was developed by rinsing the sample in AZ MIF 726 for 20 s and wash off in DI water (carried out extra times if necessary). Next, SL1 was developed using HF 7300 developer for 20 s and spin rinsed at 3000 RPM (carried out extra times if necessary). Excess of p(g3T2-T) was removed by O$_{2}$-plasma etching for 180 s. The sample was placed in Orthogonal Stripper 900 overnight at room temperature, to complete removal, ultrasonication in acetone was added for 15 min the next day. 

\begin{figure}[ht]
	\centering
	\includegraphics[width=\textwidth]{Images/pdf/undoped-patterning.pdf}
	\caption{Visualization of the workflow for patterning undoped p(g3T2-T).}
	\label{fig:undopedpat}
\end{figure}

\paragraph{Patterning doped-p(g3T2-T)}Patterning of the doped-p(g3T2-T) was achieved with photolithography, as shown in Figure \ref{fig:dopedpat}, first the orthogonal photoresist OSCoR 4020 was spin coated at 3000 RPM for 60 s and baked for another 60 s at $103^{\circ}$C. Exposure of negative resist was performed for 20 s with shadowing all areas except the ones of interest, one extra cycle was added if using higher dopant concentration (10 mg/mL). After post-baking for 60 s at $103^{\circ}$C, OSCoR was developed using Orthogonal Developer 103a for 50 s and spin rinsed at 3000 RPM for 60 s (carried out extra times if necessary). Excess of doped-p(g3T2-T) was removed by O$_{2}$-plasma etching for 180 s. The sample was placed in Orthogonal Stripper 900 overnight at room temperature.

\begin{figure}[ht]
	\centering
	\includegraphics[width=10cm]{Images/pdf/doped-patterning.pdf}
	\caption{Visualization of the workflow for patterning doped p(g3T2-T).}
	\label{fig:dopedpat}
\end{figure}

\paragraph{OECT with undoped and doped p(g3T2-T) channel.} A solid-state electrolyte precursor was prepared, as describe in Table \ref{tab:sse}, and 20$\mu$L was dropcasted to the patterned samples (undoped and doped) covering all 14 devices. Finally, a Ag/AgCl pellet was installed and used as gate (Figure \ref{fig:biosetup}). Transfer characteristics were measured under ambient conditions with different source-drain voltages (V$_{DS}$= -0.1, -0.2, -0.5V).

\begin{table}[h]
	\centering
	\caption{Composition of the solid-state electrolyte \cite{weissbachPhotopatternableSolidElectrolyte2022}.}
	\begin{tabular}{r r} \hline
		Component   & Amount \\ \hline
		H$_{2}$O	& 1.0 mL \\ 
		$[$EMIM$][$EtSO$_{4}]$   & 1.5 mL \\ 
		MBBAm   & 20 mg \\ 
		NIPAm   & 750 mg \\ 
		HHPAA   & 200 mg \\ \hline
	\end{tabular}
	\label{tab:sse}
\end{table}

\begin{figure}[!ht]
	\centering
	\includegraphics[width=7cm]{Images/pdf/bioprobe_setup.pdf}
	\caption{Experimental setup to measure threshold voltage shift under ambient conditions.}
	\label{fig:biosetup}
\end{figure}

%\newpage
\subsubsection{Stability on air of p(g3T2-T)}
In this subsection, the stability of undoped and doped samples of p(g3T2-T) (here F$_{6}$TCNNQ is being used) in N$_{2}$ and ambient conditions were tested at different stages of the photolithography process by measuring the channel conductivity under a fixed biased drain-source voltage. %, upon spin-coating, patterning, i) upon dropcast of solid-state electrolyte precursor, and after exposure, ii) after photopatterned solid-state electrolyte, and finally

\paragraph{Channel conductivity measurements.}Undoped-p(g3T2-T) were measured in N$_{2}$ and ambient conditions immediately after spin-coating onto substrate with patterned gold contacts. Then, they were measured after patterning process (as described in previous section). Finally, upon drop-casting solid-state electrolyte precursor. Doped-p(g3T2-T) channel was measured after patterning and upon drop-casting solid-state electrolyte precursor.

\subsubsection{Reverse oxidation of undoped p(g3T2-T)}

\paragraph{By electrochemical dedoping.}Undoped-p(g3T2-T) samples onto patterned-gold-contacts substrates were prepared. The solid-state-electrolyte precursor was dropcast onto the 14 devices, one device source-drain was biased (-0.3V) and applied a positive gate voltage (+1V). Channel conductivity was monitored. Finally, other devices in the sample were measured to detect any effect of the process.

\paragraph{By heating.}An undoped-p(g3T2-T) solid-OECT, whose fabrication will be described in the following subsection (\ref{subsec:solidOECT}), was exposed to ambient conditions, devices were clearly oxidized. The sample was placed in a hot plate at 120$^{\circ}$C, brought back to glovebox, and measure the following day.  %\cite{weissbachPhotopatternableSolidElectrolyte2022}.
%% alcohol rinse
\subsubsection{Solid OECTs using undoped p(g3T2-T)} \label{subsec:solidOECT}
After following the patterning steps for undoped-(pg3T2-T) from subsection \ref{subsec:photo}. Three different methods of applying the solid-state electrolyte were tested. Then, transfer characteristics at V$_{DS}$=-0.1V at a scanning rate of 0.083 V/s were measured. Cyclic voltammetry and electrochemical impedance spectroscopy were measured at fixed parameters described in section \ref{param}, one day after the fabrication.

\paragraph{OECTs with dropcast Solid-State Electrolyte.}Solid-state electrolyte precursor was dropcast and then expose for 2 cycles of 60 s in mask aligner, as shown in Figure \ref{fig:undopedsse}A). 

\paragraph{OECTs with photopatternable Solid-State Electrolyte.}First, the application of an adhesion promoter was used, composed by reactants from Table \ref{tab:adprom}, the sample was covered in Petri dish at 50$^{\circ}$C, then rinse very thoroughly with etanol and dry at least 10 min at 100$^{\circ}$C on hot plate. Solid-state-electrolyte precursor was dropcast, then a foil was place on top to avoid contact with mask that will shadow all areas except the ones of interest in the negative resist, expose for 2 cycles of 60 s in mask aligner, and blow-off with N$_{2}$ gun, as shown in Figure \ref{fig:undopedsse}B),  following reference \cite{weissbachPhotopatternableSolidElectrolyte2022}.

\begin{table}[h]
	\centering
	\caption{Composition of adhesion promoter was defined by Biosens group members at IAPP.}
	\begin{tabular}{r r} \hline
		Component   & Amount \\ \hline
		SilaneA174	& 30 $\mu$L \\ 
		Etanol   & 3mL \\ 
		Acetic acid   & 60 $\mu$L \\ \hline
	\end{tabular}
	\label{tab:adprom}
\end{table}

\paragraph{OECTs with inkjet-printed Solid-State Electrolyte.}First, the application of an adhesion promoter was used, composed by reactants from Table \ref{tab:adprom}, the sample was covered in Petri dish at 50$^{\circ}$C, then rinse very thoroughly with etanol and dry at least 10 min at 100$^{\circ}$C on hot plate. Solid-state-electrolyte precursor was ink-jet printed and expose for 2 cycles of 60 s in mask aligner, parameters and procedure was established by members of BioSens group at IAPP, as shown in Figure \ref{fig:undopedsse}C). 

\begin{figure}[!ht]
	\centering
	\includegraphics[width=\textwidth]{Images/pdf/undoped-sse.pdf}
	\caption[Solid-OECT fabrication with undoped p(g3T2-T)]{Visualization of the workflow for solid-OECT fabrication with undoped p(g3T2-T) by A) dropcasting SSE, B) photopatterning SSE, C) ink-jet printing SSE.}
	\label{fig:undopedsse}
\end{figure}

\subsubsection{Solid OECTs using doped p(g3T2-T)}
After following the patterning steps for doped-(pg3T2-T) from subsection \ref{subsec:photo}. Solid-state-electrolyte precursor was ink-jet printed and expose for 2 cycles of 60 s in mask aligner, parameters and procedure was established by members of BioSens group at IAPP. Then, transfer characteristics at V$_{DS}$=-0.1V at a scanning rate of 0.083 V/s were measured. Cyclic voltammetry and electrochemical impedance spectroscopy were measured at fixed parameters described in section \ref{param}, one day after the fabrication.



\begin{figure}[!ht]
	\centering
	\includegraphics[width=10cm]{Images/pdf/doped-sse.pdf}
	\caption[Solid-OECT fabrication with doped p(g3T2-T)]{Visualization of the workflow for solid-OECT fabrication with doped p(g3T2-T) by ink-jet printing SSE.}
	\label{fig:dopedsse}
\end{figure}

%%% Local Variables: 
%%% mode: latex
%%% TeX-master: "thesis"
%%% End: 

% ... and so on until
%\chapter{Results and Discussion}
\label{cha:3}

\section{Doped Conjugated Polymer films}
\subsection{Polaron and Bipolar formation}

\subsection{Workfunction increase}

\subsection{Redox properties with Solid State Electrolyte Precursor}


\section{Organic Electrochemical Transistors}
\subsection{Channel and gate morphology}

\subsection{Channel conductivity}

Prior biasing gate, which is due to passive (ion) diffusion?
\subsubsection{Under inert conditions}

\subsubsection{Under ambient conditions}

\subsection{Threshold voltage shift}

The Ag/AgCl gate electrode’s work function is reasonably constant, the work function of an OMIEC gate electrode however may vary depending on its processing history and redox reactions with other species present in the electrolyte (e.g. molecular oxygen).28 Applying VGS only determines the potential difference between the gate and channel but does not control the potentials of either electrode (hence the position of the Fermi level) with respect to a reference. This leads to many challenges in operating an OECT with OMIEC gate electrodes.


\subsection{Important figures of merit}


%\section{Conclusion}
%\lipsum[86-88]

%%% Local Variables: 
%%% mode: latex
%%% TeX-master: "thesis"
%%% End: 

%\chapter{Conclusions and Outlook}
\label{cha:conclusion}
%The final chapter contains the overall conclusion. It also contains suggestions for future work and industrial applications.

%Besides the challenges of working with the conjugated polymer p(g3T2-T), due to its lower HOMO level that make it susceptible to molecular oxygen in ambient conditions and its highly volatile solvent, it was possible to successfully fabricate micro-structured accumulation-mode OECTs and solid-OECTs in ambient conditions. The section devoted to Doping characterization in the Results and Discussions Chapter demonstrated that it was possible to increase doping levels by using solution sequential doping. The level of homogeneity required to allow photopatterning processes was also achieved. However, accurate control of the doping level remains a challenge, an estimation of the molecular percentage of dopants cannot be assure due to the unknown quantity of dopants that succesfully diffuse throughout the thickness of the polymer, the characterization techniques used in this work present limitations to perform a full scanning of the doping level of the material within all its volume. Certainly, there is room for improvement in trying to achieve the level of control required. 



%%% Local Variables: 
%%% mode: latex
%%% TeX-master: "thesis"
%%% End: 


% If you have appendices:
%\appendixpage*          % if wanted
%\appendix
%\chapter{Absorption of p(g3T2-T) doped with F6TCNNQ}
\label{app:A}





%%% Local Variables: 
%%% mode: latex
%%% TeX-master: "thesis"
%%% End: 

% ... and so on until
%\include{app-n}

\backmatter
% The bibliography comes after the appendices.
% You can replace the standard "abbrv" bibliography style by another one.
\bibliographystyle{IEEEtran}
%\bibliography{references}
\bibliography{mscreferences}

\end{document}

%%% Local Variables: 
%%% mode: latex
%%% TeX-master: t
%%% End: 
