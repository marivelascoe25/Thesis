\chapter{Experimental Methods} \label{cha:2}
%\lipsum[77]

\section{Materials}
%\lipsum[78]
All reactives were purchased from commercial suppliers. %and non further chemical modification or purification was done unless stated before.

\begin{itemize}
\item Chromium etchant: High purity ceric ammonium nitrate, Standard, Sigma Aldrich
\item Gold etchant:  HHPAA (2-Hydroxy-4’-(2-hydroxyethoxy)-2-methylpropiophenone), 98$\%$, Standard, Sigma Aldrich
\item Developer: AZ 726 MIF Developer, Merck performance Materials GmbH
\item EG: Ethylene glycol, $\geq$ 95$\%$, Sigma Aldrich
\item $[$EMIM$][$EtSO$_{4}]$: (1-Ethyl-3-methylimidazolium ethyl sulfate), $\geq$95$\%$, Sigma Aldrich 
\item MBBAm: (N,N’-Methylenebisacrylamide), 99$\%$, Sigma Aldrich 
\item NIPAm: (N-Isopropylacrylamide), 97$\%$, Alfa Aesar 
\item Sacrificial Layer 1: Sacrifical Layer 1 (SL1), Orthogonal Inc
\item Photoresist: AZ 1518 Photoresist, Merck Performance Materials GmbH \& Microchemical GmbH
\item Photoresist for undoped species: NLOF 2020, commercial negative-tone photoresist, Microchemical
\item Orthogonal Photoresist for doped species: OSCoR 4020 Photoresist, Orthogonal Inc.
\item Developer for SL1: Developer HF 7300, Orthogonal Inc.
\item Orthogonal Developer for OSCor 4020: Orthogonal Developer 103a, Orthogonal Inc.
\item Orthogonal Stripper: Orthogonal Stripper 900, Orthogonal Inc. 
\item p(g3T2-T): 3-(2-(2-(2-methoxyethoxy)ethoxy)ethoxy)thiophene from Professor Iain McCulloch, Chemistry Research Laboratory, University of Oxford. 
\item Dopants: 1,3,4,5,7,8-hexafluorotetracyanonaphthoquinodimethane (F$_{6}$TCNNQ) and 1,3,4,5-tetrafluorotetracyanonaphthoquinodimethane (F$_{4}$TCNQ), 97$\%$, Sigma Aldrich.
\item Adhesion promoter: Silane A174 (3-(Trimethoxysilyl)propyl methacrylate), TCI.

\end{itemize}

\section{Equipment}
\begin{itemize}
\item Baking: All baking steps were carried out on a Stuart SD160 digital hotplate (Stuart Equipment, UK). 
\item Electrical characterization (ambient): Device characterization under ambient conditions was performed on a Everbeing C-6 Probe Station (Everbeing Int’l Corp., Taiwan), connected to a Keithley 4200-SCS Semiconductor Characterisation System (Keithley Instruments, USA). 
\item Electrical characterization (glovebox): Device characterization was performed in a nitrogen-filled glovebox. Probing needles were connected to two Keithley 236 Source Measure Units (Keithley Instruments, USA). 
\item Cyclic voltammetry and Impedance measurements: Measurements were carried out by using a Metrohm Autolab PGSTAT302N potentiostat/galvanostat (Metrohm AG, Switzerland).
\item Micrographs: Micrographs were taken on a Nikon Eclipse LV100ND microscope, equipped with a DS-Fi2 camera (Nikon, Japan). 
\item Photolithography: Photolithography was carried out on a SÜSS Microtec MJB4 maskaligner system (SÜSS Microtec AG, Germany). 
\item Photomasks: Photomasks were custom made by Compugraphics Jena in a 4-inch format (soda-lime glass covered with chromium) and held several mask designs (Compugraphics Jena GmbH, Germany). 
\item Plasma cleaning: O$_{2}$-plasma cleaning was performed by using a Harrick PDC-002 plasma cleaner (Harrick Plasma, USA), connected to a Leybold Heraeus Combitron CM 330 Vacuum Gauge Controller (Leybold GmbH, Germany). 
\item Plasma etching: O$_{2}$-plasma etching was performed by using a Diener electronic ATTO plasma cleaner (Diener electronic GmbH \& Co. KG, Germany). 
\item Profilometry: Profilometry was performed on a Veeco Dektak 150 surface profiler (Veeco Instruments Inc., USA).
\item Film resistance measurements: The film resistance was measured using a linear four-point probe system (Lucas four-point probe connector) connected to a multimeter (Keithley 2010 Multimeter).
\item Transmittance and Reflectance measurements: Measurements were performed with UV-Visible-NearInfraRed Spectroscopy on a SolidSpec-3700 UV-Vis-NIR spectrometer using an integral sphere from Shimadzu.
\item Energy of HOMO/HBEC cutoff measurements: Measurements were done using Ultraviolet Photoelectron Spectroscopy (UPS) on a PHOIBOS 100 from Specs, a Helium plasma discharge lamp (UVS10/35, Specs).
\item Spincoating: Samples were coated with a SAWATEC SM-180-BT spincoater (SAWATEC AG, Switzerland).
\end{itemize}

\section{Software} \label{param}
%% Add citations of libraries and stuff
\begin{itemize}
\item Data processing: All data was processed by customized scripts written in Python. Files import and manipulation was done by using OS \cite{os_module} and CSV \cite{csv_module} modules. Mathematical computations (e.g. fits, integration) were carried out by employing the NumPy \cite{numpy_2012}, SciPy \cite{scipy_linreg}, and PeakDetect \cite{peakdetect} libraries. Visualisations were performed using the Matplotlib library \cite{matplotlib_2012}. All is compiled in the following GitHub Repository: \href{https://github.com/marivelascoe25/Thesis.git}{\textbf{marivelasco25/Thesis.git}}.
\item Electrical characterization: Measurements were performed by controlling SMUs through the in-house developed SweepMe! software \href{https://sweep-me.net/}{\textbf{(https://sweep-me.net/)}}. 
\item Profilometry: Profilometry was performed by using the Dektak software (Veeco Instruments Inc., USA).
\item Cyclic voltammetry: Measurements were performed by using the NOVA software (Metrohm AG, Switzerland). Parameters were fixed and are described in the following table: 

\begin{table}[ht]
	\centering
	\caption{Cyclic Voltammetry parameters.}
	\begin{tabular}{r r} \hline
		Parameter	& Value \\ \hline
		Start/Stop potential	& 0 V \\ 
		Upper vertex potential	& 1 V \\ 
		Lower vertex potential	& -1 V \\ 
		Number of scans	& 10 \\ 
		Scan rate	& 0.1 V/s \\ \hline
		%Step	& 0.05 \\ 
	\end{tabular}
	\label{tab:CV}
\end{table}

\item Impedance measurements: Measurements were performed by using the NOVA software (Metrohm AG, Switzerland) in potentiostatic mode. Parameters were fixed and are described in the following table: 

\begin{table}[ht]
	\centering
	\caption{Electrochemical Impedance Spectroscopy parameters.}
	\begin{tabular}{r r} \hline
		Parameter	& Value \\ \hline
		Initial frequency	& 10$^{5}$ Hz  \\ 
		Final frequency	& 10$^{-1}$ Hz \\ 
		Frequencies per decade	& 10 \\ 
		Amplitude (V$_{RMS}$)	& 0.01 V \\ \hline
	\end{tabular}
	\label{tab:EIS}
\end{table}

%105 - 10-1
%10 frequencies per decade
%Amplitude 0.01 VRMS

\end{itemize}

\section{Photomask}
The employed photolithography mask for OECT fabrication depict a specific layout of gold electrodes, as illustrated in Figure \ref{fig:mask}. Detailed information about the photolithography process is provided in the following section. 

This layout accommodates 14 devices, each with a channel length of 70 $\mu$m and a 20 $\mu$m overlap on both sides. The channel width is set at 190 $\mu$m. Additionally, the gate electrode dimensions describes a length of 190 $\mu$m and a width of 220 $\mu$m, with a gate-channel distance of 30 $\mu$m. All 14 devices will be located within a glass sample substrate measuring 2.5 $\times$ 2.5 cm$^{2}$.

\begin{figure}[ht]
  \centering
  \includegraphics[width=5cm]{Images/pdf/photomask.pdf}
  \caption{Used photomask for fabricated OECT gold electrodes.}
  \label{fig:mask}
\end{figure}

\section{Experimental Procedure}

All fabrication steps were performed under standard cleanroom conditions.
\subsection{Preparation of Films} \label{subsec:films}
\paragraph{Dynamic spin-coating of p(g3T2-T).} In contrast to the approach described in reference \cite{tanTuningOrganicElectrochemical2022}, which employed drop-casting for the deposition of undoped and doped p(g3T2-T) films, our objective is to enable photolithography and facilitate a miniaturization process. To achieve this, uniform films are needed. The BioSens group members at IAPP had previously established dynamic spin-coating method due to the high volatility of p(g3T2-T) solvent, chloroform. Substrates were cleaned through sequential steps of ultrasonic bath in acetone for 15 minutes, IPA rinse, N$_{2}$ drying, and 5 minutes of O$_{2}$ plasma etching (0.3 mbar). Then, 70$\mu$L of 10 mg/mL of p(g3T2-T) mixed at 60$^{\circ}$C for 20 minutes, was applied using dynamic spin-coating at 3000 RPM for 60s, yielding approximately 70nm thick films. The samples were then dried at 80$^{\circ}$C.

\begin{figure}[ht]
  \centering
  \includegraphics[width=8cm]{Images/pdf/doping_formulas1.pdf}
  \caption{Chemical structures of the repeat units of p(g3T2-T) and F$_{4}$TCNQ dopant.}
  \label{fig:dop1}
\end{figure}

\paragraph{Dynamic spin-coating of F$_{4}$TCNQ dopant.}Different doping levels of p(g3T2-T) were achieved by dynamic spin-coating 140$\mu$L of dopants at different concentrations (5, 10 and 20 mg/mL, as seen in Figure \ref{fig:coating}), previously mixed in acetronitrile at $60^{\circ}$C for 20 minutes. The samples were then dried at 80$^{\circ}$C.

\begin{figure}[ht]
  \centering
  \includegraphics[width=10cm]{Images/pdf/spin_coating.pdf}
  \caption{Dynamic spin-coating process to obtain undoped and doped films of p(g3T2-T)).}
  \label{fig:coating}
\end{figure}

\subsection{Doping Characterization of Films}

\paragraph{Profilometer.}The films were scratched 4 times to remove part of material, and measure the cavity depth to obtain the film thickness.

\paragraph{Four-point probe.} A four-point probe was used to calculate the relation between induced voltage in inner probes and input current between outer probes. Then sheet resistance and resistivity were calculated using the Van Der Pauw method, following reference \cite{resist}, via equations:

\begin{equation}\label{eq:rs}
	R_{S} = \frac{\pi}{ln(2)}R = 4,53 R \quad [\Omega/sq],
\end{equation}

where R is the resistance measured with the four-point probe setup, and

\begin{equation}\label{eq:resist}
	\rho = R_{s}t \quad [\Omega.cm],
\end{equation}
 
 where $t$ is the thickness of the film.

\paragraph{UV-Vis-NIR Spectroscopy.}After the preparation of the films onto quartz substrates, transmittance (T) and reflectance (R) was measured using the UV-Vis-NIR Spectrometer in the range of 285 to 1600 nm. Then, absorption (A) was calculated via

\begin{equation}\label{eq:abs}
	A = 1 - T - R,
\end{equation}

and normalized with respect to the incoming light \cite{uvvis}. Additional measurements were taken, after some days of storage under ambient conditions to check its stability in air.

\paragraph{Ultraviolet Photoelectron Spectroscopy.}After the preparation of the films, %(air exposure) 
the energy of the highest occupied molecular orbital cutoff (E$_{HOMO}$) and the high binding energy cutoff E$_{HBEC}$ was measuring using UPS. The pressure in the chamber during measurements was about 5$\cdot$10$^{-9}$ mbar, while the base pressure is in the range of 10$\cdot$10$^{-10}$ mbar. The workfunction (WF) and ionization energy (IE) are calculated by using the following equations:

\begin{equation}\label{eq:wf}
	WF = h\nu - E_{HBEC} \quad [eV],
\end{equation}

\begin{equation}\label{eq:ie}
	IE = h\nu - (E_{HBEC}-E_{HOMO}) \quad [eV],
\end{equation}

where $h\nu = 21.22 eV$, the main He I excitation line of the Helium plasma discharge lamp \cite{buchholtzDopingPropertiesNovel2021}. 
 
%\subsection{Roughness characterization of films} 

\subsection{Doping in Organic Electrochemical Transistors} \label{subsec:oect}

\subsubsection{Influence of Doping on OECT Channel} \label{subsec:channel}

Photolithographically patterned substrates with gold contacts for source, drain and gate were obtained using mask layout illustrated in Figure \ref{fig:mask}. The steps outlined in references \cite{weissbachPhotopatternableSolidElectrolyte2022} and \cite{bongartzOrganicElectrochemicalTransistors2021} were followed, which is also represented in Figure \ref{fig:aupat}.

\begin{figure}[ht]
	\centering
	\includegraphics[width=12cm]{Images/pdf/Au-patterning.pdf}
	\caption[Visualization of the workflow for Au-contacts patterning.]{Visualization of the workflow for Au-contacts patterning. Image modified based on reference \cite{bongartzOrganicElectrochemicalTransistors2021}.}
	\label{fig:aupat}
\end{figure}

Then p(g3T2-T) films were deposited following the procedure described in the previous section. The patterning of doped-p(g3T2-T) channel adopted a procedure similar to the one described in references \cite{weissbachPhotopatternableSolidElectrolyte2022}\cite{bongartzOrganicElectrochemicalTransistors2021}, which involves the use of PEDOT:PSS. However, some modifications in the exposure and developing times were made. In contrast, the patterning of undoped-p(g3T2-T) samples, required a different protocol. A sacrificial layer is introduced to ensure proper cross-linking with photoresist. Both processes are detailed in the following:

\paragraph{Patterning undoped p(g3T2-T).}Patterning of the p(g3T2-T) was achieved with photolithography, as shown in Figure \ref{fig:undopedpat}. First, a fluoropolymer Sacrificial Layer 1 (SL1) was spin-coated at 6000 RPM for 60 s, followed by a baking step for 180 s at $113^{\circ}$C. Before the deposition of photoresist, a O$_{2}$ plasma cleaning step was applied for 60 s to promote adhesion. Then, NLOF 2020 photoresist was spin-coated at 3000 RPM for 60 s and baked for another 60 s at $113^{\circ}$C. Exposure of negative resist was performed for 12 s by shadowing all areas except the ones of interest. After post-baking for 60 s at $113^{\circ}$C, NLOF was developed by rinsing the sample in AZ MIF 726 for 20 s and wash off in DI water (carried out extra times if necessary). Next, SL1 was developed using HF 7300 developer for 45 s and spin rinsed at 3000 RPM (carried out extra times if necessary). Excess of p(g3T2-T) was removed by O$_{2}$-plasma etching for 180 s. The sample was placed in Orthogonal Stripper 900 overnight at room temperature, to complete the removal. Finally, ultrasonication in acetone was added for 15 min the next day. 

\begin{figure}[ht]
	\centering
	\includegraphics[width=12.5cm]{Images/pdf/undoped-patterning.pdf}
	\caption{Visualization of the workflow for patterning undoped p(g3T2-T).}
	\label{fig:undopedpat}
\end{figure}

\paragraph{Patterning doped p(g3T2-T).}Patterning of the doped p(g3T2-T) was achieved with photolithography, as shown in Figure \ref{fig:dopedpat}. First, the orthogonal photoresist OSCoR 4020 was spin coated at 3000 RPM for 60 s and baked for another 60 s at $103^{\circ}$C. Exposure of negative resist was performed for 20 s with shadowing all areas except the ones of interest, one extra cycle was added if using higher dopant concentration (10 mg/mL). After post-baking for 60 s at $103^{\circ}$C, OSCoR was developed using Orthogonal Developer 103a for 45 s and spin rinsed at 3000 RPM for 60 s (carried out extra times if necessary). Excess of doped-p(g3T2-T) was removed by O$_{2}$-plasma etching for 180 s. The sample was placed in Orthogonal Stripper 900 overnight at room temperature.

\begin{figure}[ht]
	\centering
	\includegraphics[width=10.5cm]{Images/pdf/doped-patterning.pdf}
	\caption{Visualization of the workflow for patterning doped p(g3T2-T).}
	\label{fig:dopedpat}
\end{figure}

\paragraph{OECT with undoped- and doped-p(g3T2-T) channel.} A solid-state electrolyte precursor was prepared according to the details provided in Table \ref{tab:sse}. Next, 20$\mu$L of this precursor were drop-casted to the patterned samples, both undoped and doped, covering all 14 devices in the sample. Finally, a Ag/AgCl pellet was installed and used as gate (Figure \ref{fig:biosetup}). Transfer characteristics were measured under ambient conditions with a V$_{GS}$ swept backwards from 1.0 V to -0.8 V. %at a scanning rate of 0.083 V/s.

\begin{table}[ht]
	\centering
	\caption{Composition of the solid-state electrolyte \cite{weissbachPhotopatternableSolidElectrolyte2022}.}
	\begin{tabular}{r c l} \hline
		Component   & Amount & Function \\ \hline
		H$_{2}$O	& 1.0 mL & dilution \\ 
		$[$EMIM$][$EtSO$_{4}]$   & 1.5 mL & ionic liquid \\ 
		MBBAm   & 20 mg & crosslinker \\ 
		NIPAm   & 750 mg & monomer \\ 
		HHPAA   & 200 mg & photoinitiator \\
		EG	& 1.5 mL	& increase viscosity, ensures good print \\ 
		Triton & 1 drop & surfactant, ensures good print \\  \hline
	\end{tabular}
	\label{tab:sse}
\end{table}

\begin{figure}[!ht]
	\centering
	\includegraphics[width=5.5cm]{Images/pdf/bioprobe_setup.pdf}
	\caption{Representation of experimental setup for measuring threshold voltage shift of a single OECT under ambient conditions. Scheme is not drawn to scale.}
	\label{fig:biosetup}
\end{figure}

\newpage
%\subsubsection{Stability on Air of p(g3T2-T)}
\subsubsection{Stability of Undoped and Doped p(g3T2-T)}
In this subsection, stability tests were conducted on both undoped and doped p(g3T2-T) samples. These samples were doped with F$_{6}$TCNNQ, following same procedure described in subsection \ref{subsec:films}. %The channel conductivity was measured %at various stages of the photolithography process, 
%before and after patterning and after contact with SSE %
%while a fixed biased drain-source voltage was maintained, under N$_{2}$ and ambient conditions. %, upon spin-coating, patterning, i) upon dropcast of solid-state electrolyte precursor, and after exposure, ii) after photopatterned solid-state electrolyte, and finally

\begin{figure}[ht]
  \centering
  \includegraphics[width=7cm]{Images/pdf/doping_formulas2.pdf}
  \caption{Chemical structures of the repeat units of p(g3T2-T) and F$_{6}$TCNNQ dopant.}
  \label{fig:dop2}
\end{figure}

\paragraph{Channel conductivity measurements.}The channel conductivity of an undoped-p(g3T2-T) device was measured in N$_{2}$ then ambient conditions immediately after spin-coating onto substrate with patterned gold contacts. The channel conductivities of both doped and undoped devices were monitored after patterning process (as described in previous section) for two hours under N$_{2}$ environment, and for 1h30 under ambient environment. Finally, they were again monitored upon drop-casting solid-state electrolyte precursor.
%Undoped-p(g3T2-T) sample was measured in N$_{2}$ then ambient conditions immediately after spin-coating onto substrate with patterned gold contacts. Then, sample was measured after patterning process (as described in previous section). Finally, upon drop-casting solid-state electrolyte precursor. The doped-p(g3T2-T) channel was measured after patterning and upon drop-casting solid-state electrolyte precursor.

\paragraph{Transfer characteristics.} Transfer characteristics were measured at different $V_{DS}$ (-0.5, -0.3, -0.1 V) of the undoped p(g3T2-T) device using a Ag/AgCl pellet as gate under environmental conditions, immediately after finalizing the process described above.

\subsubsection{Counteracting Oxidation of Undoped p(g3T2-T) by Electrochemical De-doping}

\paragraph{Channel conductivity measurements.}Undoped p(g3T2-T) samples onto patterned-gold-contacts substrates were prepared. The solid-state electrolyte precursor was drop-casted onto the 14 devices, one device source-drain was negatively-biased and applied a positive gate voltage (+1 V). Channel conductivity was monitored. %Finally, other devices on the sample were measured to detect any effect of the process.

\paragraph{Transfer characteristics.}Transfer characteristics were measured at $V_{DS}$ of -0.1 using the OMIEC gate under N$_{2}$ environment, immediately after process described above. Other measurements were performed after UV light exposure, and in ambient conditions using a Ag/AgCl pellet gate.
 
%\paragraph{By heating.}An undoped-p(g3T2-T) solid-OECT, whose fabrication will be described in the following subsection \ref{subsec:solidOECT}, was exposed to ambient conditions, devices were clearly oxidized. The sample was placed on a hot plate at 120$^{\circ}$C, brought back to glovebox, and measured the following day.  %\cite{weissbachPhotopatternableSolidElectrolyte2022}.
%% alcohol rinse

\subsection{Fabrication of Solid Organic Electrochemical Transistors} \label{subsec:soect}

\subsubsection{Solid-OECTs using Undoped-p(g3T2-T)} \label{subsec:solidOECT}
After following the patterning steps for undoped-p(g3T2-T) from Subsection \ref{subsec:channel}. Three different methods of applying the solid-state electrolyte precursor were tested. Then, transfer characteristics were measured at a constant V$_{DS}$ of -0.1V, and a  V$_{GS}$ swept backwards from 1.0 V to 1.0 V at a scanning rate of 0.083 V/s. Cyclic voltammetry and electrochemical impedance spectroscopy were performed by short-circuiting source and drain to probe between channel and gate. Parameters were fixed as described in Section \ref{param}, and measurements were taken one day after the fabrication.

\paragraph{Application of adhesion promoter.}An adhesion promoter, consisting of reactants detailed in Table \ref{tab:adprom}, was applied to the substrate. The sample was placed in a Petri dish at 50$^{\circ}$C and covered for 20 min. Subsequently, it was thoroughly rinse with ethanol and dried on a hot plate at 100$^{\circ}$C for a minimum of 10 min. The composition and application steps of adhesion promoter was previously established by Biosens group members at IAPP. 

\begin{table}[ht]
	\centering
	\caption{Composition of adhesion promoter.}
	\begin{tabular}{r r} \hline
		Component   & Amount \\ \hline
		SilaneA174	& 30 $\mu$L \\ 
		Etanol   & 3mL \\ 
		Acetic acid   & 60 $\mu$L \\ \hline
	\end{tabular}
	\label{tab:adprom}
\end{table}

\paragraph{OECT with drop-casted Solid-State Electrolyte.}Solid-state electrolyte precursor was drop-casted and then exposed for 2 cycles of 60 s in a mask aligner, as shown in Figure \ref{fig:undopedsse}a. No adhesion promoter was applied in this sample.

\paragraph{OECT with photopatternable Solid-State Electrolyte.}After the application of the adhesion promoter, the solid-state electrolyte precursor was drop-casted onto the sample, followed by the placement of a Teflon foil to prevent contact with the mask, which would shadow all areas except the ones of interest in the negative resist. The sample was then exposed for 2 cycles of 60 s in a mask aligner. Removal of excess, non-cross-linked precursor was careful blown off with a N$_{2}$ gun, as shown in Figure \ref{fig:undopedsse}b,  following references \cite{weissbachPhotopatternableSolidElectrolyte2022}\cite{bongartzOrganicElectrochemicalTransistors2021}.

\paragraph{OECT with inkjet-printed Solid-State Electrolyte.}After the application of the adhesion promoter, the solid-state-electrolyte precursor was ink-jet printed and expose for 2 cycles of 60 s in mask aligner, as illustrated in Figure \ref{fig:undopedsse}c. Parameters and procedure for ink-jet printing was previously established by members of BioSens group at IAPP \cite{tsengThresholdVoltageControl2023}. 

\begin{figure}[!ht]
	\centering
	\includegraphics[width=12cm]{Images/pdf/undoped-sse.pdf}
	\caption[Solid-OECT fabrication with undoped p(g3T2-T)]{Visualization of the workflow for solid-OECT fabrication with undoped p(g3T2-T) by A) drop-casting SSE, B) photopatterning SSE, C) ink-jet printing SSE.}
	\label{fig:undopedsse}
\end{figure}

\subsubsection{Solid-OECTs using Doped-p(g3T2-T)}

\paragraph{OECT with inkjet-printed Solid-State Electrolyte.}After the deposition of films and dopant (F$_{6}$TCNNQ), films were stored in glovebox over the weekend to let dopants soak and dry. Then, the patterning steps for doped-p(g3T2-T) from Subsection \ref{subsec:channel} were followed with no extra cycles in exposure times. Solid-state electrolyte precursor was ink-jet printed as described in the previous subsection. Then, transfer characteristics were measured at a constant V$_{DS}$ of -0.1V, and  a  V$_{GS}$ swept backwards from 1.0 V to 1.0 V at a scanning rate of 0.083 V/s. Cyclic voltammetry and electrochemical impedance spectroscopy were performed by short-circuiting source and drain to probe between channel and gate. Parameters were fixed as described in Section \ref{param}, and measurements were taken one day after the fabrication.

\begin{figure}[!t]
	\centering
	\includegraphics[width=9cm]{Images/pdf/doped-sse.pdf}
	\caption[Solid-OECT fabrication with doped p(g3T2-T)]{Visualization of the workflow for solid-OECT fabrication with doped p(g3T2-T) by ink-jet printing SSE.}
	\label{fig:dopedsse}
\end{figure}

%%% Local Variables: 
%%% mode: latex
%%% TeX-master: "thesis"
%%% End: 
